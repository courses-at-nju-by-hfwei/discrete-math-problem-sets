% hw2-predicate-logic-solution.tex

% !TEX program = xelatex
%%%%%%%%%%%%%%%%%%%%
% see http://mirrors.concertpass.com/tex-archive/macros/latex/contrib/tufte-latex/sample-handout.pdf
% for how to use tufte-handout
\documentclass[a4paper, justified]{tufte-handout}

% hw-preamble.tex

% geometry for A4 paper
% See https://tex.stackexchange.com/a/119912/23098
\geometry{
  left=20.0mm,
  top=20.0mm,
  bottom=20.0mm,
  textwidth=130mm, % main text block
  marginparsep=5.0mm, % gutter between main text block and margin notes
  marginparwidth=50.0mm % width of margin notes
}

% for colors
\usepackage{xcolor} % usage: \color{red}{text}
% predefined colors
\newcommand{\red}[1]{\textcolor{red}{#1}} % usage: \red{text}
\newcommand{\blue}[1]{\textcolor{blue}{#1}}
\newcommand{\teal}[1]{\textcolor{teal}{#1}}

\usepackage{todonotes}

% heading
\usepackage{sectsty}
\setcounter{secnumdepth}{2}
\allsectionsfont{\centering\huge\rmfamily}

% for Chinese
\usepackage{xeCJK}
\usepackage{zhnumber}
\setCJKmainfont[BoldFont=FandolSong-Bold.otf]{FandolSong-Regular.otf}

% for fonts
\usepackage{fontspec}
\newcommand{\song}{\CJKfamily{song}}
\newcommand{\kai}{\CJKfamily{kai}}

% To fix the ``MakeTextLowerCase'' bug:
% See https://github.com/Tufte-LaTeX/tufte-latex/issues/64#issuecomment-78572017
% Set up the spacing using fontspec features
\renewcommand\allcapsspacing[1]{{\addfontfeature{LetterSpace=15}#1}}
\renewcommand\smallcapsspacing[1]{{\addfontfeature{LetterSpace=10}#1}}

% for url
\usepackage{hyperref}
\hypersetup{colorlinks = true,
  linkcolor = teal,
  urlcolor  = teal,
  citecolor = blue,
  anchorcolor = blue}

\newcommand{\me}[4]{
    \author{
      {\bfseries 姓名:}\underline{#1}\hspace{2em}
      {\bfseries 学号:}\underline{#2}\hspace{2em}\\[10pt]
      {\bfseries 评分:}\underline{#3\hspace{3em}}\hspace{2em}
      {\bfseries 评阅:}\underline{#4\hspace{3em}}
  }
}

% Please ALWAYS Keep This.
\newcommand{\noplagiarism}{
  \begin{center}
    \fbox{\begin{tabular}{@{}c@{}}
      请独立完成作业,不得抄袭。\\
      若得到他人帮助, 请致谢。\\
      若参考了其它资料,请给出引用。\\
      鼓励讨论,但需独立书写解题过程。
    \end{tabular}}
  \end{center}
}

% \newcommand{\goal}[1]{
%   \begin{center}{\fcolorbox{blue}{yellow!60}{\parbox{0.50\textwidth}{\large
%     \begin{itemize}
%       \item 体会``思维的乐趣''
%       \item 初步了解递归与数学归纳法
%       \item 初步接触算法概念与问题下界概念
%     \end{itemize}}}}
%   \end{center}
% }

% Each hw consists of four parts:
\newcommand{\beginrequired}{\hspace{5em}\section{作业 (必做部分)}}
\newcommand{\beginoptional}{\section{作业 (选做部分)}}
\newcommand{\beginot}{\section{Open Topics}}
\newcommand{\begincorrection}{\section{订正}}
\newcommand{\beginfb}{\section{反馈}}

% for math
\usepackage{amsmath, mathtools, amsfonts, amssymb}
\newcommand{\set}[1]{\{#1\}}

% define theorem-like environments
\usepackage[amsmath, thmmarks]{ntheorem}

\theoremstyle{break}
\theorempreskip{2.0\topsep}
\theorembodyfont{\song}
\theoremseparator{}
\newtheorem{problem}{题目}[subsection]
\renewcommand{\theproblem}{\arabic{problem}}
\newtheorem{definition}{定义}[subsection]
\renewcommand{\thedefinition}{\arabic{definition}}
\newtheorem{lemma}{引理}[subsection]
\renewcommand{\thelemma}{\arabic{lemma}}
\newtheorem{ot}{Open Topics}

\theorempreskip{3.0\topsep}
\theoremheaderfont{\kai\bfseries}
\theoremseparator{:}
\theorempostwork{\bigskip\hrule}
\newtheorem*{solution}{解答}
\theorempostwork{\bigskip\hrule}
\newtheorem*{revision}{订正}

\theoremstyle{plain}
\newtheorem*{cause}{错因分析}
\newtheorem*{remark}{注}

\theoremstyle{break}
\theorempostwork{\bigskip\hrule}
\theoremsymbol{\ensuremath{\Box}}
\newtheorem*{proof}{证明}

% \newcommand{\ot}{\blue{\bf [OT]}}

% for figs
\renewcommand\figurename{图}
\renewcommand\tablename{表}

% for fig without caption: #1: width/size; #2: fig file
\newcommand{\fig}[2]{
  \begin{figure}[htbp]
    \centering
    \includegraphics[#1]{#2}
  \end{figure}
}
% for fig with caption: #1: width/size; #2: fig file; #3: caption
\newcommand{\figcap}[3]{
  \begin{figure}[htbp]
    \centering
    \includegraphics[#1]{#2}
    \caption{#3}
  \end{figure}
}
% for fig with both caption and label: #1: width/size; #2: fig file; #3: caption; #4: label
\newcommand{\figcaplbl}[4]{
  \begin{figure}[htbp]
    \centering
    \includegraphics[#1]{#2}
    \caption{#3}
    \label{#4}
  \end{figure}
}
% for margin fig without caption: #1: width/size; #2: fig file
\newcommand{\mfig}[2]{
  \begin{marginfigure}
    \centering
    \includegraphics[#1]{#2}
  \end{marginfigure}
}
% for margin fig with caption: #1: width/size; #2: fig file; #3: caption
\newcommand{\mfigcap}[3]{
  \begin{marginfigure}
    \centering
    \includegraphics[#1]{#2}
    \caption{#3}
  \end{marginfigure}
}

\usepackage{fancyvrb}

% for algorithms
\usepackage[]{algorithm}
\usepackage[]{algpseudocode} % noend
% See [Adjust the indentation whithin the algorithmicx-package when a line is broken](https://tex.stackexchange.com/a/68540/23098)
\newcommand{\algparbox}[1]{\parbox[t]{\dimexpr\linewidth-\algorithmicindent}{#1\strut}}
\newcommand{\hStatex}[0]{\vspace{5pt}}
\makeatletter
\newlength{\trianglerightwidth}
\settowidth{\trianglerightwidth}{$\triangleright$~}
\algnewcommand{\LineComment}[1]{\Statex \hskip\ALG@thistlm \(\triangleright\) #1}
\algnewcommand{\LineCommentCont}[1]{\Statex \hskip\ALG@thistlm%
  \parbox[t]{\dimexpr\linewidth-\ALG@thistlm}{\hangindent=\trianglerightwidth \hangafter=1 \strut$\triangleright$ #1\strut}}
\makeatother

% for footnote/marginnote
% see https://tex.stackexchange.com/a/133265/23098
\usepackage{tikz}
\newcommand{\circled}[1]{%
  \tikz[baseline=(char.base)]
  \node [draw, circle, inner sep = 0.5pt, font = \tiny, minimum size = 8pt] (char) {#1};
}
\renewcommand\thefootnote{\protect\circled{\arabic{footnote}}}

\newcommand{\score}[1]{{\bf [#1 分]}} % feel free to modify this file if you understand LaTeX well
%%%%%%%%%%%%%%%%%%%%
\title{2. 一阶谓词逻辑 (2-predicate-logic)}
\me{魏恒峰}{hfwei@nju.edu.cn}{}{}
\date{2021年03月18日 发布习题 \\ 2021年04月02日 发布答案}
%%%%%%%%%%%%%%%%%%%%
\begin{document}
\maketitle
%%%%%%%%%%%%%%%%%%%%
\noplagiarism % PLEASE DON'T DELETE THIS LINE!
%%%%%%%%%%%%%%%%%%%%
\begin{abstract}
  \mfig{width = 1.00\textwidth}{figs/logic-invincible}
  % \mfigcap{width = 0.85\textwidth}{figs/George-Boole}{George Boole}
  % \begin{center}{\fcolorbox{blue}{yellow!60}{\parbox{0.65\textwidth}{\large
  %   \begin{itemize}
  %     \item
  %   \end{itemize}}}}
  % \end{center}
\end{abstract}
%%%%%%%%%%%%%%%%%%%%
\beginrequired
%%%%%%%%%%%%%%%

%%%%%%%%%%%%%%%
\begin{problem}[命题逻辑: 形式化描述与推理 \score{3} $\star\star$]
  张三说李四在说谎, 李四说王五在说谎, 王五说张三、李四都在说谎。
  请问, 这三人到底谁在说真话, 谁在说谎?
  \noindent {(要求: 需给出关键的推理步骤或理由)}
\end{problem}

\begin{solution}
  我们使用 $Z$、$L$、$W$ 分别表示``张三在说真话''、``李四在说真话''与``王五在说真话''。
  题目中的三个论断可以形式化为~\footnote{注意: 这里是 $\leftrightarrow$, 不止是 $\to$。}:
  \begin{gather}
    Z \leftrightarrow \lnot L \label{eq:1} \\
    L \leftrightarrow \lnot W \label{eq:2} \\
    W \leftrightarrow \lnot Z \land \lnot L \label{eq:3}
  \end{gather}
  在如下推理中, 我们分别考虑 $Z$、$L$、$W$ 成立的情况。
  如果某种情况可以推出矛盾 ($\bot$), 则说明相应情况不成立。
  \begin{itemize}
    \item \textsc{Case I}: 引入假设 $[Z]$, 可推出矛盾, 故 $\lnot Z$。
      \begin{align}
        [Z] &\quad (\text{引入假设}) \label{eq:4} \\
        \lnot L &\quad ((\ref{eq:1}), (\ref{eq:4})) \label{eq:5} \\
        W &\quad ((\ref{eq:2}), (\ref{eq:5})) \label{eq:6} \\
        \lnot Z \land \lnot L &\quad ((\ref{eq:3}), (\ref{eq:6})) \label{eq:7} \\
        \lnot Z &\quad (\land\text{-elim}, (\ref{eq:7})) \label{eq:8} \\
        \bot &\quad ((\ref{eq:4}), (\ref{eq:8}))
      \end{align}
    \item \textsc{Case II}: 类似于 \textsc{Case I},
      引入假设 $[W]$, 同样可推出矛盾, 故 $\lnot W$。
    \item \textsc{Case III}: 引入假设 $L$, 无矛盾, 得出 $\lnot Z \land \lnot W \land L$。
      \setcounter{equation}{3}
      \begin{align}
        [L] &\quad (\text{引入假设}) \label{eq:4} \\
        \lnot Z &\quad ((\ref{eq:1}), (\ref{eq:4})) \label{eq:5} \\
        \lnot W &\quad ((\ref{eq:2}), (\ref{eq:4})) \label{eq:6} \\
        \lnot(\lnot Z \land \lnot L) &\quad ((\ref{eq:3}), (\ref{eq:6})) \label{eq:7} \\
        Z \lor L &\quad ((\ref{eq:7})) \label{eq:8} \\
      \end{align}
  \end{itemize}
  综上, 有 $\lnot Z$, $\lnot W$, $L$。
\end{solution}
%%%%%%%%%%%%%%%

%%%%%%%%%%%%%%%
\begin{problem}[一阶谓词逻辑: 形式化描述与推理\score{3} $\star\star$]
  给定如下``前提'', 请判断``结论''是否有效, 并说明理由。
  请使用一阶谓词逻辑的知识解答。
  \noindent {(要求: 需给出关键的推理步骤或理由)} \\[10pt]

  {\bf 前提:}
  \begin{enumerate}[(1)]
    \item 每个人或者喜欢美剧, 或者喜欢韩剧 (可以同时喜欢二者);
    \item 任何人如果他喜欢抗日神剧, 他就不喜欢美剧;
    \item 有的人不喜欢韩剧。
  \end{enumerate}

  \vspace{0.20cm}
  \noindent {\bf 结论:} 有的人不喜欢抗日神剧 ({\it 幸亏如此})。
\end{problem}

\begin{solution}
  定义谓词:
  \begin{itemize}
    \item $A(x)$: $x$ 喜欢美剧;
    \item $K(x)$: $x$ 喜欢韩剧;
    \item $J(x)$: $x$ 喜欢抗日神剧。
  \end{itemize}
  题目中的三个论断可以形式化为:
  \begin{itemize}
    \item $\forall x.\; A(x) \lor K(x)$;
    \item $\forall x.\; J(x) \to \lnot A(x)$;
    \item $\exists x.\; \lnot K(x)$
  \end{itemize}
  先示范纯形式化推理:
  \setcounter{equation}{0}
  \begin{align}
    \forall x.\; A(x) \lor K(x) & \quad (\text{前提})
      \label{eq:1} \\[6pt]
    \forall x.\; J(x) \to \lnot A(x) & \quad (\text{前提})
      \label{eq:2} \\[6pt]
    \exists x.\; \lnot K(x) & \quad (\text{前提})
      \label{eq:3} \\[6pt]
    \red{[x_{0}] \quad [\lnot K(x_{0})]} & \quad (\text{引入变量与假设})
      \label{eq:4} \\[6pt]
    A(x_{0}) \lor K(x_{0}) & \quad (\forall\text{-elim}, (\ref{eq:1}), (\ref{eq:4}))
      \label{eq:5} \\[6pt]
    A(x_{0}) & \quad ((\ref{eq:4}), (\ref{eq:5}))
      \label{eq:6} \\[6pt]
    J(x_{0}) \to \lnot A(x_{0}) & \quad (\forall\text{-elim}, (\ref{eq:2}), (\ref{eq:4}))
      \label{eq:7} \\[6pt]
    \lnot J(x_{0}) & \quad ((\ref{eq:6}), (\ref{eq:7}))
      \label{eq:8} \\[6pt]
    \red{\exists x.\; \lnot J(x)} & \quad (\exists\text{-intro}, (\ref{eq:8}))
      \label{eq:9} \\[6pt]
    \exists x.\; \lnot J(x) & \quad (\exists\text{-elim}, (\ref{eq:3})-(\ref{eq:8}))
      \label{eq:10}
  \end{align}
  \marginnote{
    \begin{description}
      \item[问:] 第 (\ref{eq:4}) 步中的 $\lnot K(x_{0})$ 为什么是个假设,
        而不是根据第 (\ref{eq:3}) 步推导出来的一个命题?
      \item[答:] 当前提里有 $\exists x.\; P(x)$ 时, 我们通常会在证明中这样做:
        \red{假设} $P(x)$ 对 $x_{0}$ 成立, 然后基于此进行推导。
        第 (\ref{eq:4}) 步正是在做这件事。
    \end{description}
  }

  \marginnote{
    \begin{description}
      \item[问:] 第 (\ref{eq:9}) 步已经得到了 $\exists x.\; \lnot J(x)$,
        为什么不``见好就收''呢?
      \item[答:] 第 (\ref{eq:9}) 步是在第 (\ref{eq:4}) 步中引入的假设 $[\lnot K(x_{0})]$
        的基础上得到的, 此时该假设还没有释放。到第 (\ref{eq:9}) 步, 我们完成了 $\exists\text{-elim}$
        规则的横线上方的部分, 第 (\ref{eq:10}) 步便可以运用 $\exists\text{-elim}$ 规则得到
        横线下方的结论, 这样假设也被释放了。
        (在平时作业与考试中, 我们也可以不用这么严格, 参见下面一种书写格式。)
    \end{description}
  }

  在作业与考试时, 只要推理清晰, 不一定要严格遵守纯形式推理的要求。
  以下也是可以接受的书写过程~\footnote{公式编号与每条公式的推理依据要写清楚}:

  \setcounter{equation}{3}
  \noindent 根据 (\ref{eq:3}), 不妨设 $\lnot K(x)$ 对 $x_{0}$ 成立:
  \begin{align}
    \lnot K(x_{0}) \label{eq:4}
  \end{align}
  根据 (\ref{eq:1}), 有
  \begin{align}
    A(x_{0}) \lor K(x_{0}) \label{eq:5}
  \end{align}
  根据 (\ref{eq:4}) 与 (\ref{eq:5}), 有
  \begin{align}
    A(x_{0}) \label{eq:6}
  \end{align}
  根据 (\ref{eq:2}), 有
  \begin{align}
    J(x_{0}) \to \lnot A(x_{0}) \label{eq:7}
  \end{align}
  根据 (\ref{eq:6}) 与 (\ref{eq:7}), 有
  \begin{align}
    \lnot J(x_{0}) \label{eq:8}
  \end{align}
  因此, $\exists x.\; \lnot J(x)$ 成立。

  作为对比, 我们再给出一种无法接受的解答: \\
  \lipsum[1]
  因此, 易见 $\exists x.\; \lnot J(x)$ 成立。
\end{solution}
%%%%%%%%%%%%%%%

%%%%%%%%%%%%%%%
\begin{problem}[一阶谓词逻辑: 形式化描述与推理 \score{4} $\star\star$]
  请使用一阶谓词逻辑公式描述以下两个定义,
  并从逻辑推理的角度说明这两种定义之间是否有强弱之分。
  \noindent {(要求: 需给出关键的推理步骤或理由)}

  A function $f$ from $\mathbb{R}$ to $\mathbb{R}$ is called
  \begin{enumerate}[(1)]
    \setlength{\itemsep}{10pt}
    \item \blue{\it pointwise continuous} (连续的) if
      for every $x \in \mathbb{R}$
      and every real number $\epsilon > 0$,
      there exists real $\delta > 0$ such that
      for every $y \in \mathbb{R}$ with $|x - y| < \delta$,
      we have that $|f(x) -  f(y)|< \epsilon$.
    \item \blue{\it uniformly continuous} (一致连续的) if
      for every real number $\epsilon > 0$,
      there exists real $\delta > 0$ such that
      for every $x, y \in \mathbb{R}$ with $|x - y| < \delta$,
      we have that $|f(x) -  f(y)|< \epsilon$.
  \end{enumerate}
\end{problem}

\begin{solution}
  形式化表示如下:
  \setcounter{equation}{0}
  \begin{enumerate}[(1)]
    \item
      \begin{align}
        \forall x \in \mathbb{R}.\; \forall \epsilon \in \mathbb{R}^{+}.\;
          \exists \delta \in \mathbb{R}^{+}.\;
          \forall y \in \mathbb{R}.\; |x - y| < \delta \to |f(x) - f(y)| < \epsilon.
        \label{eq:cont}
      \end{align}

      \marginnote{
        为了简便, 也可以统一将论域限制在实数集 $\mathbb{R}$ 上。
        这样就可以将公式写成
        \begin{align*}
          &\forall x.\; \forall \epsilon > 0.\; \exists \delta > 0.\; \forall y.\; \\
            &\quad |x - y| < \delta \to |f(x) - f(y)| < \epsilon.
        \end{align*}
      }

      \marginnote{
        \begin{description}
          \item[问:] $\forall \epsilon \in \mathbb{R} \land \epsilon > 0.\; (\dots)$ 是否符合语法?
          \item[答:] 我也不能确定。一般来说, 这种简记法要尽可能简短。
            可以写成 $\forall \epsilon \in \mathbb{R}^{+}$。
        \end{description}
      }
    \item
      \begin{align}
        \forall \epsilon \in \mathbb{R}^{+}.\; \exists \delta \in \mathbb{R}^{+}.\;
          \forall x \in \mathbb{R}.\; \forall y \in \mathbb{R}.\; |x - y| < \delta \to |f(x) - f(y)| < \epsilon.
        \label{eq:uniform-cont}
      \end{align}
      \marginnote{
        ``$\forall x \in \mathbb{R}.\; \forall y \in \mathbb{R}.$''
        也可以简写成 ``$\forall x, y \in \mathbb{R}.$''。
        注意, 必须是一样的量词符号才可以这样简写。
      }
  \end{enumerate}
  下面从逻辑的角度讨论 (\ref{eq:cont}) 与 (\ref{eq:uniform-cont}) 的强弱关系。

  首先, 根据重言式
  \[
    \forall x.\; \forall y.\; \alpha \leftrightarrow \forall y.\; \forall x.\; \alpha,
  \]
  (\ref{eq:cont}) 等价于
  \begin{align*}
    \forall \epsilon \in \mathbb{R}^{+}.\; \forall x \in \mathbb{R}.\;
      \exists \delta \in \mathbb{R}^{+}.\;
      \forall y \in \mathbb{R}.\; |x - y| < \delta \to |f(x) - f(y)| < \epsilon.
  \end{align*}

  其次, 根据重言式
  \[
    \exists x.\; \forall y.\; \alpha \to \forall y.\; \exists x.\; \alpha,
  \]
  可知 (\ref{eq:uniform-cont}) \red{不弱于} (\ref{eq:cont})。

  最后, 由于~\footnote{思考: 为什么? 举例说明。}
  \[
    \forall y.\; \exists x.\; \alpha \nvdash \exists x.\; \forall y.\; \alpha,
  \]
  可知 (\ref{eq:uniform-cont}) (严格)强于 (\ref{eq:cont})。

  作为例子, $f(x) = x^{2}$ 是连续函数, 但不是一致连续函数。
\end{solution}
%%%%%%%%%%%%%%%

%%%%%%%%%%%%%%%%%%%%
% 如果没有需要订正的题目,可以把这部分删掉

\begincorrection
%%%%%%%%%%%%%%%%%%%%

%%%%%%%%%%%%%%%%%%%%
% 如果没有反馈,可以把这部分删掉
\beginfb

你可以写 (也可以发邮件或者使用``教学立方'')
\begin{itemize}
  \item 对课程及教师的建议与意见
  \item 教材中不理解的内容
  \item 希望深入了解的内容
  \item $\cdots$
\end{itemize}
%%%%%%%%%%%%%%%%%%%%
\end{document}