% problems.tex

\documentclass{article}
\usepackage{xeCJK}
\usepackage{amsmath}
\usepackage{hyperref}

\title{离散数学 习题集}
\author{魏恒峰}
\date{\today}
%%%%%%%%%%%%%%%%%%%%%%%%%%%%%%
\begin{document}
\maketitle
%%%%%%%%%%%%%%%%%%%%
\section{参考书籍}

\begin{itemize}
  \item 绿皮书:《离散数学 理论、分析、题解》(左孝凌 等; 上海科学技术文献出版社)
  \item 红皮书:《离散数学》(左孝凌 等; 上海科学技术文献出版社)
\end{itemize}
%%%%%%%%%%%%%%%%%%%%
\section{命题逻辑 (2021年03月28日)}

以下内容参见``绿皮书'':
\begin{itemize}
  \item 表 1-1 (P7)、表 1-2 (P7)
  \item 例题 1-1 (P8)、例题 1-2 (P9)、例题 1-5 (P15)、例题 1-6 (P16)
  \item 习题 1-3 (P20)
  \item 习题 1-14 (P24)
  \item 习题 1-20 (P31)
  \item 习题 1-23 (P33)
  \item 习题 1-44 (P42)
  \item 习题 1-55 (P51)
  \item 习题 1-58 (P56)
  \item 习题 1-60 (P58)
\end{itemize}

说明:
\begin{itemize}
  \item 符号用法: 该习题集中的 $\implies$ 相当于我们的 $\models$ (重言蕴含);
    $\iff$ 相当于我们的 $\equiv$ (重言等价)
  \item 本习题集中的推理可以使用任何(见表1-1, 1-2)蕴含式与重言式所表示的推理规则。
    参见 P6 中的 ``$T$ 规则''。
  \item 若有疑问, 可到 \href{https://github.com/courses-at-nju-by-hfwei/discrete-math-problem-sets/discussions}{problem-sets@github} 提问讨论。
\end{itemize}

%%%%%%%%%%%%%%%%%%%%
\section{一阶谓词逻辑 (待定)}

\end{document}
%%%%%%%%%%%%%%%%%%%%%%%%%%%%%%