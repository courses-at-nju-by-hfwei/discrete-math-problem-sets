% problems.tex

% !TEX program = xelatex
\documentclass{article}
\usepackage{xeCJK}
\usepackage{xcolor}
\usepackage{amsmath, amssymb}
\usepackage{hyperref}

\title{离散数学 习题集}
\author{魏恒峰}
\date{\today}
%%%%%%%%%%%%%%%%%%%%%%%%%%%%%%
\begin{document}
\maketitle
%%%%%%%%%%%%%%%%%%%%
\section*{参考书籍}

\begin{itemize}
  \item 绿皮书:《离散数学 理论、分析、题解》(左孝凌 等; 上海科学技术文献出版社)
  \item 红皮书:《离散数学》(左孝凌 等; 上海科学技术文献出版社)
  \item《离散数学及其应用》(第8版; By Rosen)
  \item《离散数学结构》(第6版; By Kolman)
\end{itemize}
%%%%%%%%%%%%%%%%%%%%
\section{命题逻辑 (2021年03月28日)}

以下内容参见``绿皮书'' (共 12 道题目):
\begin{itemize}
  \item 表 1-1 (P7)、表 1-2 (P7)
  \item 例题 1-1 (P8)、例题 1-2 (P9)、例题 1-5 (P15)、例题 1-6 (P16)
  \item 习题 1-3 (P20)
  \item 习题 1-14 (P24)
  \item 习题 1-20 (P31)
  \item 习题 1-23 (P33)
  \item 习题 1-44 (P42)
  \item 习题 1-55 (P51)
  \item 习题 1-58 (P56)
  \item 习题 1-60 (P58)
\end{itemize}

说明:
\begin{itemize}
  \item 符号用法: 该习题集中的 $\implies$ 相当于我们的 $\models$ (重言蕴含);
    $\iff$ 相当于我们的 $\equiv$ (重言等价)
  \item 本习题集中的推理可以使用任何(见表1-1, 1-2)蕴含式与重言式所表示的推理规则。
    参见 P6 中的 ``$T$ 规则''。
  \item 若有疑问, 可到 \href{https://github.com/courses-at-nju-by-hfwei/discrete-math-problem-sets/discussions}{problem-sets@github} 提问讨论。
\end{itemize}

%%%%%%%%%%%%%%%%%%%%
\section{一阶谓词逻辑 (2021年03月29日)}

以下内容参见``绿皮书'' (共 12 道题目):
\begin{itemize}
  \item 表 2-1 (P69)
  \item 例 2-1 (P69)、例 2-3 (P71)、例 2-4 (P72)
  \item 习题 2-2 (P75)、习题 2-10 (P79)
  \item 习题 2-11 (P80)、习题 2-15 (P82)、习题 2-20 (P84)
  \item 习题 2-27 (P89)、习题 2-28 (P91)、习题 2-29 (P91)、习题 2-30 (P93)
\end{itemize}

说明:
\begin{itemize}
  \item 本习题集中在 $\forall x$ 前后加括号, 写作 $(\forall x)$。我们不需要加。
  \item 若有疑问, 可到 \href{https://github.com/courses-at-nju-by-hfwei/discrete-math-problem-sets/discussions}{problem-sets@github} 提问讨论。
\end{itemize}
%%%%%%%%%%%%%%%%%%%%
\section{数学归纳法 (2021年04月02日)}

本节没有找到带有详细解答的习题集。
下面介绍两本书, 有兴趣的同学可以看一看。
\begin{itemize}
  \item《具体数学》第一章: P1 $\sim$ P12
  \item《数学归纳法》(华罗庚)。中学生数学课外读物, 可以先快速浏览, 然后集中看一些对自己来说有些难度的地方。
\end{itemize}
%%%%%%%%%%%%%%%%%%%%
\section{集合论: 基本概念与运算 (2021年04月02日)}

以下内容参见``绿皮书'' (共 15 道题目):
\begin{itemize}
  \item 习题 3-1 (P119)、习题 3-8 (P121)
  \item 习题 3-15 (P123)、习题 3-16 (P123)
  \item 习题 3-25 (P126)、习题 3-26 (P127)、习题 3-27 (P127)
  \item 习题 3-34 (P132)、习题 3-35 (P132)、习题 3-36 (P132)、习题 3-37 (P132)
  \item 习题 3-38 (P133)、习题 3-39 (P133)
  \item 习题 3-45 (P135)、习题 3-48 (P135)
\end{itemize}

说明:
\begin{itemize}
  \item 本习题集虽然有详细的解答, 但是有些解答的格式不够简洁, 也不够美观。
    在阅读这些解答时, 请思考你会如何用更优雅的方式书写答案~\footnote{关于证明, 形式与内容同样重要。}
  \item \textcolor{red}{注: 习题 3-45 答案有误。}
  \item 若有疑问, 可到 \href{https://github.com/courses-at-nju-by-hfwei/discrete-math-problem-sets/discussions}{problem-sets@github} 提问讨论。
\end{itemize}
%%%%%%%%%%%%%%%%%%%%
\section{集合论: 关系 (I) (2021年04月08日)}

以下内容参见``绿皮书'' (共 13 道题目):
\begin{itemize}
  \item 习题 3-58 (P144)、习题 3-66 (P148)
  \item 习题 3-68 (P149)、习题 3-69 (P149)
  \item 习题 3-71 (P151)、习题 3-73 (P151)、
  \item 习题 3-77 (P153)、习题 3-81 (P154)
  \item 习题 3-87 (P158)、习题 3-89 (P158)、习题 3-91 (P160)
  \item 习题 3-112 (P173)、习题 3-114 (P174)
\end{itemize}

说明:
\begin{itemize}
  \item 若有疑问, 可到 \href{https://github.com/courses-at-nju-by-hfwei/discrete-math-problem-sets/discussions}{problem-sets@github} 提问讨论。
\end{itemize}
%%%%%%%%%%%%%%%%%%%%
\section{集合论: 函数 (2021年04月18日)}

以下内容参见``绿皮书'' (共 16 道题目):
\begin{itemize}
  \item 习题 4-1 (P208)、习题 4-4 (P208)
  \item 习题 4-8 (P210)、习题 4-12 (P211)
  \item 习题 4-16 (P213)
  \item 习题 4-25 (P217)、习题 4-27 (P218)
  \item 习题 4-28 (P219)、习题 4-29 (P219)
  \item 习题 4-32 (P221)、习题 4-33 (P221)
  \item 习题 4-34 (P222)、习题 4-37 (P223)
  \item 习题 4-40 (P225)、习题 4-42 (P225)、习题 4-43 (P226)
\end{itemize}

说明:
\begin{itemize}
  \item 本习题集分别使用 $f \ast X$ 与 $f^{-1} \ast Y$ 表示 $f(X)$ 与 $f^{-1}(Y)$。
  \item 本习题集的``入射''表示``单射''。
  \item 本次习题包含了我们在下次课(2021-04-22)要介绍的关于函数的内容。
  \item 若有疑问, 可到 \href{https://github.com/courses-at-nju-by-hfwei/discrete-math-problem-sets/discussions}{problem-sets@github} 提问讨论。
\end{itemize}
%%%%%%%%%%%%%%%%%%%%
\section{集合论: 序关系 (2021年04月23日)}

以下内容参见``绿皮书'' (共7道题目):
\begin{itemize}
  \item 习题 3-141 (P189)
  \item 习题 3-142 (P190)、习题 3-142 (P190)
  \item 习题 3-144 (P192)、习题 3-146 (P193)
  \item 习题 3-148 (P195)、习题 3-149 (P196)
\end{itemize}

说明:
\begin{itemize}
  \item \url{https://www.bilibili.com/video/BV1EJ411V7a4?share_source=copy_web}
  \item 本次习题包含了我们在下次课(2021-04-29)要介绍的关于序关系的内容。
  \item 若有疑问, 可到 \href{https://github.com/courses-at-nju-by-hfwei/discrete-math-problem-sets/discussions}{problem-sets@github} 提问讨论。
\end{itemize}
%%%%%%%%%%%%%%%%%%%%
\section{集合论: 无穷 (2021年05月02日)}

以下内容参见``绿皮书'' (共9道题目):
\begin{itemize}
  \item 例题 4-5 (P206)
  \item 习题 4-61 (P234)、习题 4-62 (P234)、习题 4-64 (P235)
  \item 习题 4-69 (P238)、习题 4-72 (P239)
  \item 习题 4-76 (P241)
  \item 习题 4-78 (P242; 可选)、习题 4-79 (P242; 可选)
\end{itemize}

说明:
\begin{itemize}
  \item 本习题集使用 $A \sim B$ 表示 $A$ 与 $B$ 等势。
  \item 本习题集使用 $\aleph$ 表示实数集 $\mathbb{R}$ 的势。
  \item 若有疑问, 可到 \href{https://github.com/courses-at-nju-by-hfwei/discrete-math-problem-sets/discussions}{problem-sets@github} 提问讨论。
\end{itemize}
%%%%%%%%%%%%%%%%%%%%
\section{图论: 路径与圈 (2021年05月10日)}

以下内容参见``绿皮书'' (共18道题目):
\begin{itemize}
  \item 例题 7-1 (P358)、例题 7-2 (P358)、例题 7-6 (P361)
  \item 习题 7-2 (P369)、习题 7-12 (P375)、习题 7-13 (P375)、习题 7-16 (P377)
  \item 习题 7-37 (P394)、习题 7-38 (P394)、习题 7-39 (P395)、习题 7-40 (P395)、
  习题 7-41 (P397)
  \item 习题 7-42 (P399)、习题 7-43 (P399)、习题 7-44 (P400)、习题 7-46 (P401)
  \item 习题 7-49 (P403)、习题 7-51 (P404)
\end{itemize}

说明:
\begin{itemize}
  \item 本习题集使用 ``路''表示``道路 (walk)'', 使用 ``通路'' 表示 ``路径 (path)''。
  \item 若有疑问, 可到 \href{https://github.com/courses-at-nju-by-hfwei/discrete-math-problem-sets/discussions}{problem-sets@github} 提问讨论。
\end{itemize}
%%%%%%%%%%%%%%%%%%%%
\end{document}
%%%%%%%%%%%%%%%%%%%%%%%%%%%%%%