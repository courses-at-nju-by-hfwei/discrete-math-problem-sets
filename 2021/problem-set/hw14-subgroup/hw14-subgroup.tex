% hw14-subgroup.tex

% !TEX program = xelatex
%%%%%%%%%%%%%%%%%%%%
% see http://mirrors.concertpass.com/tex-archive/macros/latex/contrib/tufte-latex/sample-handout.pdf
% for how to use tufte-handout
\documentclass[a4paper, justified]{tufte-handout}

\input{hw-preamble} % feel free to modify this file if you understand LaTeX well
%%%%%%%%%%%%%%%%%%%%
\title{14. 群论: 子群 (14-subgroup)}
\me{魏恒峰}{hfwei@nju.edu.cn}{}{}
\date{2021年6月11日}
%%%%%%%%%%%%%%%%%%%%
\begin{document}
\maketitle
%%%%%%%%%%%%%%%%%%%%
\noplagiarism % PLEASE DON'T DELETE THIS LINE!
%%%%%%%%%%%%%%%%%%%%
\begin{abstract}
\end{abstract}
%%%%%%%%%%%%%%%%%%%%
\beginrequired
%%%%%%%%%%%%%%%

%%%%%%%%%%%%%%%
\begin{problem}[\score{4} $\star\star$]
  设 $H \le G$。请证明,
  \[
    aH = H \iff a \in H \iff aH \le G
  \]
\end{problem}

\begin{proof}
\end{proof}
%%%%%%%%%%%%%%%

%%%%%%%%%%%%%%%
\begin{problem}[\score{5 = 2 + 3} $\star\star\star$]
  设 $\phi$ 是从群 $G$ 到 $G'$ 的同态映射。
  请证明,
  \begin{enumerate}[(1)]
    \item
      \[
        H \le G \implies \phi(H) \le G'.
      \]
    \item
      \[
        H \triangleleft G \implies \phi(H) \triangleleft G'.
      \]
  \end{enumerate}
\end{problem}

\begin{proof}
\end{proof}
%%%%%%%%%%%%%%%

%%%%%%%%%%%%%%%
\begin{problem}[\score{3} $\star\star$]
  请计算
  \[
    \begin{pmatrix}
      1 & 2 & 3 & 4 & 5 \\
      1 & 3 & 4 & 5 & 2
    \end{pmatrix}
    \begin{pmatrix}
      1 & 2 & 3 & 4 & 5 \\
      3 & 2 & 4 & 1 & 5
    \end{pmatrix},
  \]
  并将结果写成(不相交)轮换的乘积。
\end{problem}

\begin{solution}
\end{solution}
%%%%%%%%%%%%%%%

%%%%%%%%%%%%%%%
\begin{problem}[\score{3} $\star\star\star$]
  考虑如下定义。
  \begin{definition}[元素的阶]
    设 $G$ 是有限群, $e$ 为 $G$ 的单位元, $a \in G$。
    使 $a^{r} = e$ 成立的最小正整数称为 $a$ 的阶
    (order)~\footnote{注意, 群的阶指的是集合 $G$ 的大小, 即 $|G|$。},
    记作 $\text{ord}\; a = r$。
  \end{definition}
  设 $G$ 是有限群。请证明,
  \[
    \forall a \in G.\; (\text{ord}\; a) \big\vert |G|.
  \]
\end{problem}

\begin{proof}
\end{proof}
%%%%%%%%%%%%%%%

%%%%%%%%%%%%%%%
\begin{problem}[\score{5 = 2 + 1 + 2} $\star\star\star$]
  考虑从乘法群 $\R^{\ast} = \R \setminus \set{0}$
  到乘法群 $\R^{+}$ 的函数 $f: x \mapsto |x|$。
  \begin{enumerate}[(1)]
    \item 请证明, $f$ 是从 $\R^{\ast}$ 到 $\R^{+}$ 的同态。
    \item 求 $\text{Ker}\; \phi$。
    \item 套用群同态基本定理, 给出相应结论, 并用一两句话解释该结论。
  \end{enumerate}
\end{problem}

\begin{proof}
\end{proof}
%%%%%%%%%%%%%%%

%%%%%%%%%%%%%%%%%%%%
% 如果没有需要订正的题目,可以把这部分删掉
\begincorrection
%%%%%%%%%%%%%%%%%%%%

%%%%%%%%%%%%%%%%%%%%
% 如果没有反馈,可以把这部分删掉
\beginfb

你可以写 (也可以发邮件或者使用``教学立方'')
\begin{itemize}
  \item 对课程及教师的建议与意见
  \item 教材中不理解的内容
  \item 希望深入了解的内容
  \item $\cdots$
\end{itemize}
%%%%%%%%%%%%%%%%%%%%
\end{document}