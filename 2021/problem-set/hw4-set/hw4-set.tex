% hw4-set.tex

% !TEX program = xelatex
%%%%%%%%%%%%%%%%%%%%
% see http://mirrors.concertpass.com/tex-archive/macros/latex/contrib/tufte-latex/sample-handout.pdf
% for how to use tufte-handout
\documentclass[a4paper, justified]{tufte-handout}

\input{hw-preamble} % feel free to modify this file if you understand LaTeX well
%%%%%%%%%%%%%%%%%%%%
\title{4. 集合: 基本概念与运算 (4-set)}
\me{魏恒峰}{hfwei@nju.edu.cn}{}{}
\date{2021年4月01日}
%%%%%%%%%%%%%%%%%%%%
\begin{document}
\maketitle
%%%%%%%%%%%%%%%%%%%%
\noplagiarism % PLEASE DON'T DELETE THIS LINE!
%%%%%%%%%%%%%%%%%%%%
\begin{abstract}
  % \mfigcap{width = 0.85\textwidth}{figs/George-Boole}{George Boole}
  % \begin{center}{\fcolorbox{blue}{yellow!60}{\parbox{0.65\textwidth}{\large
  %   \begin{itemize}
  %     \item
  %   \end{itemize}}}}
  % \end{center}
\end{abstract}
%%%%%%%%%%%%%%%%%%%%
\beginrequired
%%%%%%%%%%%%%%%

%%%%%%%%%%%%%%%
\begin{problem}[相对补与绝对补 \score{5} $\star\star$]
  请证明,
  \[
    A \cap (B \setminus C) = (A \cap B) \setminus C
                            = (A \cap B) \setminus (A \cap C).
  \]
\end{problem}

\begin{proof}
\end{proof}
%%%%%%%%%%%%%%%

%%%%%%%%%%%%%%%
\begin{problem}[对称差 \score{4} $\star\star$]
  请证明,
  \[
    A \cap (B \oplus C) = (A \cap B) \oplus (A \cap C).
  \]
\end{problem}

\begin{proof}
\end{proof}
%%%%%%%%%%%%%%%

%%%%%%%%%%%%%%%
\begin{problem}[广义并、广义交 \score{4} $\star\star$]
  请证明,
  \[
    \mathcal{F} \cap \mathcal{G} \neq \emptyset \implies
      \bigcap \mathcal{F} \cap \bigcap \mathcal{G} \subseteq \bigcap (\mathcal{F} \cap \mathcal{G}).
  \]
  并举例说明, $\subseteq$ 不能换成 $=$。
\end{problem}

\begin{proof}
\end{proof}
%%%%%%%%%%%%%%%

%%%%%%%%%%%%%%%
\begin{problem}[广义并、广义交、德摩根律 \score{3} $\star\star\star$]
  请化简集合 $A$:
  \[
    A = \mathbb{R} \setminus \bigcap_{n \in \mathbb{Z}^{+}} (\mathbb{R} \setminus \set{-n, -n+1, \cdots, 0, \cdots, n-1, n})
  \]
\end{problem}

\begin{solution}
\end{solution}
%%%%%%%%%%%%%%%

%%%%%%%%%%%%%%%
\begin{problem}[幂集 \score{4} $\star\star\star$]
  请证明,~\footnote{不, 我有``幂集''恐惧症。}
  \[
    \ps{A} = \ps{B} \iff A = B.
  \]
\end{problem}

\begin{solution}
\end{solution}
%%%%%%%%%%%%%%%

%%%%%%%%%%%%%%%%%%%%
% 如果没有需要订正的题目,可以把这部分删掉

\begincorrection
%%%%%%%%%%%%%%%%%%%%

%%%%%%%%%%%%%%%%%%%%
% 如果没有反馈,可以把这部分删掉
\beginfb

你可以写 (也可以发邮件或者使用``教学立方'')
\begin{itemize}
  \item 对课程及教师的建议与意见
  \item 教材中不理解的内容
  \item 希望深入了解的内容
  \item $\cdots$
\end{itemize}
%%%%%%%%%%%%%%%%%%%%
\end{document}