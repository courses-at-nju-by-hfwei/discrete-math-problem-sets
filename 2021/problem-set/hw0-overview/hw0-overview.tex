% hw0-overview.tex

% !TEX program = xelatex
%%%%%%%%%%%%%%%%%%%%
% see http://mirrors.concertpass.com/tex-archive/macros/latex/contrib/tufte-latex/sample-handout.pdf
% for how to use tufte-handout
\documentclass[a4paper, justified]{tufte-handout}

\input{hw-preamble} % feel free to modify this file if you understand LaTeX well
%%%%%%%%%%%%%%%%%%%%
\title{离散数学 (0-Overview)}
\me{魏恒峰}{hfwei@nju.edu.cn}{}{}
\date{2021年3月4日}
%%%%%%%%%%%%%%%%%%%%
\begin{document}
\maketitle
%%%%%%%%%%%%%%%%%%%%
\noplagiarism % PLEASE DON'T DELETE THIS LINE!
%%%%%%%%%%%%%%%%%%%%
\begin{abstract}
  % \mfigcap{width = 0.85\textwidth}{figs/George-Boole}{George Boole}
  % \begin{center}{\fcolorbox{blue}{yellow!60}{\parbox{0.65\textwidth}{\large
  %   \begin{itemize}
  %     \item
  %   \end{itemize}}}}
  % \end{center}
\end{abstract}
%%%%%%%%%%%%%%%%%%%%
\beginrequired
%%%%%%%%%%%%%%%
\begin{problem}[防疫工作, 不能大意 \score{4}]
  近期突发一种流感,症状及其严重,受感染的学生会无可遏制地进行编程与刷题等危险行为。
  假设学生坐在$n \times n$排座位的教室里。
  感染正在迅速扩散:
  \begin{itemize}
    \item 如果某学生已被感染,那么他/她就不可能痊愈了;
    \item 如果某学生至少与2个已经感染的学生座位相邻 (前、后、左、右;不包括对角),
      那么该学生也会被感染。
  \end{itemize}
  请证明: 如果初始状态有$< n$个学生感染了流感,那么至少有一个学生永远不会被感染。
\end{problem}

\begin{solution}
\end{solution}
%%%%%%%%%%%%%%%

%%%%%%%%%%%%%%%
\begin{problem}[Nim Game \score{6 = 1 + 2 + 2 + 1}]
  Nim 是一个双人游戏 (你可以在课堂上分享的 Ludii Player 里找到它)。
  游戏开始时,两人面前放着几堆石头,
  两个玩家轮流操作,每次选择从某个石堆里拿走一块或多块石头。
  最后没有石头可拿的那个玩家输掉比赛。

  \fig{width = 0.40\textwidth}{figs/nim-stone}

  本题将引导大家寻找该游戏的必胜策略。

  考虑对石头堆里的石头个数做(二进制表示下的; 不足时高位补0)异或操作($\oplus$),
  结果称为 Nim 和。

  \begin{enumerate}[(1)]
    \item 请证明: 若 Nim 和为0, 则任意一次移动都会导致 Nim 和不为 0。
    \item 请证明: 若 Nim 和不为0,则必然存在一个石头堆,它的石头数大于其它所有石头堆的 Nim 和。
      (统一在二进制或十进制下进行比较)
    \item 请证明: 若游戏开始时, Nim 和不为 0,则先手有必胜策略。
    \item 当 $n = 2$ 时,给出某玩家有必胜策略的充要条件与他/她的必胜策略。
  \end{enumerate}
\end{problem}

\begin{solution}
\end{solution}
%%%%%%%%%%%%%%%

%%%%%%%%%%%%%%%%%%%%
% 如果没有需要订正的题目,可以把这部分删掉

\begincorrection
%%%%%%%%%%%%%%%%%%%%

%%%%%%%%%%%%%%%%%%%%
% 如果没有反馈,可以把这部分删掉
\beginfb

你可以写 (也可以发邮件或者使用``教学立方'')
\begin{itemize}
  \item 对课程及教师的建议与意见
  \item 教材中不理解的内容
  \item 希望深入了解的内容
  \item $\cdots$
\end{itemize}
%%%%%%%%%%%%%%%%%%%%
\end{document}