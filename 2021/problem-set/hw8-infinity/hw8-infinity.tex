% hw8-infinity.tex

% !TEX program = xelatex
%%%%%%%%%%%%%%%%%%%%
% see http://mirrors.concertpass.com/tex-archive/macros/latex/contrib/tufte-latex/sample-handout.pdf
% for how to use tufte-handout
\documentclass[a4paper, justified]{tufte-handout}

\input{hw-preamble} % feel free to modify this file if you understand LaTeX well
%%%%%%%%%%%%%%%%%%%%
\title{8. 集合: 无穷 (8-infinity)}
\me{魏恒峰}{hfwei@nju.edu.cn}{}{}
\date{2021年4月29日}
%%%%%%%%%%%%%%%%%%%%
\begin{document}
\maketitle
%%%%%%%%%%%%%%%%%%%%
\noplagiarism % PLEASE DON'T DELETE THIS LINE!
%%%%%%%%%%%%%%%%%%%%
\begin{abstract}
  % \mfigcap{width = 0.85\textwidth}{figs/George-Boole}{George Boole}
  % \begin{center}{\fcolorbox{blue}{yellow!60}{\parbox{0.65\textwidth}{\large
  %   \begin{itemize}
  %     \item
  %   \end{itemize}}}}
  % \end{center}
\end{abstract}
%%%%%%%%%%%%%%%%%%%%
\beginrequired
%%%%%%%%%%%%%%%

%%%%%%%%%%%%%%%
\begin{problem}[\score{5} $\star\star\star$]
  请证明鸽笼原理。
\end{problem}

\begin{proof}
\end{proof}
%%%%%%%%%%%%%%%

%%%%%%%%%%%%%%%
\begin{problem}[\score{5} $\star\star\star$]
  Is the set of all infinite sequences of $0$'s and $1$'s finite,
  countably infinite, or uncountable?
\end{problem}

\begin{proof}
\end{proof}
%%%%%%%%%%%%%%%

%%%%%%%%%%%%%%%
\begin{problem}[\score{5} $\star\star\star$]
  Give an example, if possible, of
  \begin{enumerate}[(a)]
    \item a countably infinite collection of \blue{\it pairwise disjoint} nonempty sets whose union is finite.
    \item a countably infinite collection of nonempty sets whose union is finite.
  \end{enumerate}
\end{problem}

\begin{proof}
\end{proof}
%%%%%%%%%%%%%%%

%%%%%%%%%%%%%%%
\begin{problem}[\score{5} $\star\star\star\star$]
  \begin{theorem}[Cantor-Schr\"{o}der–Bernstein (1887)]
    \[
      |X| \le |Y| \land |Y| \le |X| \implies |X| = |Y|
    \]
    \[
      \exists\; f: X \xrightarrow{1-1} Y \land g: Y \xrightarrow{1-1} X
      \implies \exists\; h: X \xleftrightarrow[\text{onto}]{1-1} Y
    \]
  \end{theorem}

  {\href{https://en.wikipedia.org/wiki/Schr\%C3\%B6der\%E2\%80\%93Bernstein\_theorem}{\teal{\footnotesize Schr\"{o}der–Bernstein theorem @ wiki}}}
\end{problem}

\begin{proof}
\end{proof}
%%%%%%%%%%%%%%%
%%%%%%%%%%%%%%%%%%%%
% 如果没有需要订正的题目,可以把这部分删掉

\begincorrection
%%%%%%%%%%%%%%%%%%%%

%%%%%%%%%%%%%%%%%%%%
% 如果没有反馈,可以把这部分删掉
\beginfb

你可以写 (也可以发邮件或者使用``教学立方'')
\begin{itemize}
  \item 对课程及教师的建议与意见
  \item 教材中不理解的内容
  \item 希望深入了解的内容
  \item $\cdots$
\end{itemize}
%%%%%%%%%%%%%%%%%%%%
\end{document}