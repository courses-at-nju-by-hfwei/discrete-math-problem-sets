% hw14-subgroup-solution.tex

% !TEX program = xelatex
%%%%%%%%%%%%%%%%%%%%
% see http://mirrors.concertpass.com/tex-archive/macros/latex/contrib/tufte-latex/sample-handout.pdf
% for how to use tufte-handout
\documentclass[a4paper, justified]{tufte-handout}

% hw-preamble.tex

% geometry for A4 paper
% See https://tex.stackexchange.com/a/119912/23098
\geometry{
  left=20.0mm,
  top=20.0mm,
  bottom=20.0mm,
  textwidth=130mm, % main text block
  marginparsep=5.0mm, % gutter between main text block and margin notes
  marginparwidth=50.0mm % width of margin notes
}

% for colors
\usepackage{xcolor} % usage: \color{red}{text}
% predefined colors
\newcommand{\red}[1]{\textcolor{red}{#1}} % usage: \red{text}
\newcommand{\blue}[1]{\textcolor{blue}{#1}}
\newcommand{\teal}[1]{\textcolor{teal}{#1}}

\usepackage{todonotes}

% heading
\usepackage{sectsty}
\setcounter{secnumdepth}{2}
\allsectionsfont{\centering\huge\rmfamily}

% for Chinese
\usepackage{xeCJK}
\usepackage{zhnumber}
\setCJKmainfont[BoldFont=FandolSong-Bold.otf]{FandolSong-Regular.otf}

% for fonts
\usepackage{fontspec}
\newcommand{\song}{\CJKfamily{song}}
\newcommand{\kai}{\CJKfamily{kai}}

% To fix the ``MakeTextLowerCase'' bug:
% See https://github.com/Tufte-LaTeX/tufte-latex/issues/64#issuecomment-78572017
% Set up the spacing using fontspec features
\renewcommand\allcapsspacing[1]{{\addfontfeature{LetterSpace=15}#1}}
\renewcommand\smallcapsspacing[1]{{\addfontfeature{LetterSpace=10}#1}}

% for url
\usepackage{hyperref}
\hypersetup{colorlinks = true,
  linkcolor = teal,
  urlcolor  = teal,
  citecolor = blue,
  anchorcolor = blue}

\newcommand{\me}[4]{
    \author{
      {\bfseries 姓名:}\underline{#1}\hspace{2em}
      {\bfseries 学号:}\underline{#2}\hspace{2em}\\[10pt]
      {\bfseries 评分:}\underline{#3\hspace{3em}}\hspace{2em}
      {\bfseries 评阅:}\underline{#4\hspace{3em}}
  }
}

% Please ALWAYS Keep This.
\newcommand{\noplagiarism}{
  \begin{center}
    \fbox{\begin{tabular}{@{}c@{}}
      请独立完成作业,不得抄袭。\\
      若得到他人帮助, 请致谢。\\
      若参考了其它资料,请给出引用。\\
      鼓励讨论,但需独立书写解题过程。
    \end{tabular}}
  \end{center}
}

% \newcommand{\goal}[1]{
%   \begin{center}{\fcolorbox{blue}{yellow!60}{\parbox{0.50\textwidth}{\large
%     \begin{itemize}
%       \item 体会``思维的乐趣''
%       \item 初步了解递归与数学归纳法
%       \item 初步接触算法概念与问题下界概念
%     \end{itemize}}}}
%   \end{center}
% }

% Each hw consists of four parts:
\newcommand{\beginrequired}{\hspace{5em}\section{作业 (必做部分)}}
\newcommand{\beginoptional}{\section{作业 (选做部分)}}
\newcommand{\beginot}{\section{Open Topics}}
\newcommand{\begincorrection}{\section{订正}}
\newcommand{\beginfb}{\section{反馈}}

% for math
\usepackage{amsmath, mathtools, amsfonts, amssymb}
\newcommand{\set}[1]{\{#1\}}

% define theorem-like environments
\usepackage[amsmath, thmmarks]{ntheorem}

\theoremstyle{break}
\theorempreskip{2.0\topsep}
\theorembodyfont{\song}
\theoremseparator{}
\newtheorem{problem}{题目}[subsection]
\renewcommand{\theproblem}{\arabic{problem}}
\newtheorem{definition}{定义}[subsection]
\renewcommand{\thedefinition}{\arabic{definition}}
\newtheorem{lemma}{引理}[subsection]
\renewcommand{\thelemma}{\arabic{lemma}}
\newtheorem{ot}{Open Topics}

\theorempreskip{3.0\topsep}
\theoremheaderfont{\kai\bfseries}
\theoremseparator{:}
\theorempostwork{\bigskip\hrule}
\newtheorem*{solution}{解答}
\theorempostwork{\bigskip\hrule}
\newtheorem*{revision}{订正}

\theoremstyle{plain}
\newtheorem*{cause}{错因分析}
\newtheorem*{remark}{注}

\theoremstyle{break}
\theorempostwork{\bigskip\hrule}
\theoremsymbol{\ensuremath{\Box}}
\newtheorem*{proof}{证明}

% \newcommand{\ot}{\blue{\bf [OT]}}

% for figs
\renewcommand\figurename{图}
\renewcommand\tablename{表}

% for fig without caption: #1: width/size; #2: fig file
\newcommand{\fig}[2]{
  \begin{figure}[htbp]
    \centering
    \includegraphics[#1]{#2}
  \end{figure}
}
% for fig with caption: #1: width/size; #2: fig file; #3: caption
\newcommand{\figcap}[3]{
  \begin{figure}[htbp]
    \centering
    \includegraphics[#1]{#2}
    \caption{#3}
  \end{figure}
}
% for fig with both caption and label: #1: width/size; #2: fig file; #3: caption; #4: label
\newcommand{\figcaplbl}[4]{
  \begin{figure}[htbp]
    \centering
    \includegraphics[#1]{#2}
    \caption{#3}
    \label{#4}
  \end{figure}
}
% for margin fig without caption: #1: width/size; #2: fig file
\newcommand{\mfig}[2]{
  \begin{marginfigure}
    \centering
    \includegraphics[#1]{#2}
  \end{marginfigure}
}
% for margin fig with caption: #1: width/size; #2: fig file; #3: caption
\newcommand{\mfigcap}[3]{
  \begin{marginfigure}
    \centering
    \includegraphics[#1]{#2}
    \caption{#3}
  \end{marginfigure}
}

\usepackage{fancyvrb}

% for algorithms
\usepackage[]{algorithm}
\usepackage[]{algpseudocode} % noend
% See [Adjust the indentation whithin the algorithmicx-package when a line is broken](https://tex.stackexchange.com/a/68540/23098)
\newcommand{\algparbox}[1]{\parbox[t]{\dimexpr\linewidth-\algorithmicindent}{#1\strut}}
\newcommand{\hStatex}[0]{\vspace{5pt}}
\makeatletter
\newlength{\trianglerightwidth}
\settowidth{\trianglerightwidth}{$\triangleright$~}
\algnewcommand{\LineComment}[1]{\Statex \hskip\ALG@thistlm \(\triangleright\) #1}
\algnewcommand{\LineCommentCont}[1]{\Statex \hskip\ALG@thistlm%
  \parbox[t]{\dimexpr\linewidth-\ALG@thistlm}{\hangindent=\trianglerightwidth \hangafter=1 \strut$\triangleright$ #1\strut}}
\makeatother

% for footnote/marginnote
% see https://tex.stackexchange.com/a/133265/23098
\usepackage{tikz}
\newcommand{\circled}[1]{%
  \tikz[baseline=(char.base)]
  \node [draw, circle, inner sep = 0.5pt, font = \tiny, minimum size = 8pt] (char) {#1};
}
\renewcommand\thefootnote{\protect\circled{\arabic{footnote}}}

\newcommand{\score}[1]{{\bf [#1 分]}} % feel free to modify this file if you understand LaTeX well
%%%%%%%%%%%%%%%%%%%%
\title{14. 群论: 子群 (14-subgroup)}
\me{魏恒峰}{hfwei@nju.edu.cn}{}{}
\date{2021年06月11日 发布作业 \\ 2021年06月23日 发布答案}
%%%%%%%%%%%%%%%%%%%%
\begin{document}
\maketitle
%%%%%%%%%%%%%%%%%%%%
\noplagiarism % PLEASE DON'T DELETE THIS LINE!
%%%%%%%%%%%%%%%%%%%%
\begin{abstract}
\end{abstract}
%%%%%%%%%%%%%%%%%%%%
\beginrequired
%%%%%%%%%%%%%%%

%%%%%%%%%%%%%%%
\begin{problem}[\score{4} $\star\star$]
  设 $H \le G$。请证明,
  \[
    aH = H \iff a \in H \iff aH \le G
  \]
\end{problem}

\begin{proof}
  \begin{enumerate}[(1)]
    \item 先证明 $aH = H \iff a \in H$。
      \begin{itemize}
        \item 先证明 $aH = H \implies a \in H$。\\
          假设 $aH = H$。
          \[
            a = ae \in aH = H.
          \]
        \item 再证明 $a \in H \implies aH = H$。\\
          假设 $a \in H$。
          首先, 由于 $H \le G$ 满足封闭性, 所以 $aH \subseteq H$。
          其次, 对于任意 $h \in H$, 由于 $H \le G$ 满足封闭性,
          所以 $a^{-1}h \in H$ 且
          \[
            h = a (a^{-1}h) \in aH.
          \]
          因此, $H \subseteq aH$。
      \end{itemize}
    \item 再证明 $a \in H \iff aH \le G$。\\
      \begin{itemize}
        \item 先证明 $a \in H \implies aH \le G$。\\
          根据 (1),
          \[
            a \in H \implies aH = H \implies aH \le G.
          \]
        \item 再证明 $aH \le G \implies a \in H$。
          \[
            aH \le G \implies e \in aH \implies a^{-1} \in H \implies a \in H.
          \]
      \end{itemize}
  \end{enumerate}
\end{proof}
%%%%%%%%%%%%%%%

%%%%%%%%%%%%%%%
\begin{problem}[\score{5 = 2 + 3} $\star\star\star$]
  设 $\phi$ 是从群 $G$ 到 $G'$ 的同态映射。
  请证明,
  \begin{enumerate}[(1)]
    \item
      \[
        H \le G \implies \phi(H) \le G'.
      \]
    \item
      \[
        H \triangleleft G \implies \phi(H) \triangleleft \phi(G).
      \]
  \end{enumerate}
\end{problem}

\begin{proof}
  \begin{enumerate}[(1)]
    \item 记 $G'$ 的单位元为 $e'$。
      $e' \in \phi(H)$, 所以 $\phi(H) \neq \emptyset$。
      因为 $H \le G$,
      所以对任意的 $h_{1}, h_{2} \in H$, 有
      \[
        h_{1}h_{2}^{-1} \in H.
      \]
      因此,
      \[
        \phi(h_{1})(\phi(h_{2})^{-1}) = \phi(h_{1})\phi(h_{2}^{-1})
          = \phi(h_{1}h_{2}^{-1}) \in \phi(H).
      \]
      所以, $\phi(H) \le G'$。
    \item 由 (1) 知,
      \[
        H \subseteq G \land \phi(H) \le G'.
      \]
      故,
      \[
        \phi(H) \le \phi(G).
      \]
      对任意 $g \in G$, $h \in H$, 因为 $H \triangleleft G$,
      \[
        ghg^{-1} \in H.
      \]
      因此,
      \[
        \phi(g)\phi(h)(\phi(g))^{-1} = \phi(g)\phi(h)\phi(g^{-1})
          = \phi(ghg^{-1}) \in \phi(H).
      \]
      所以, $\phi(H) \triangleleft \phi(G)$。
  \end{enumerate}
\end{proof}
%%%%%%%%%%%%%%%

%%%%%%%%%%%%%%%
\begin{problem}[\score{3} $\star\star$]
  请计算
  \[
    \begin{pmatrix}
      1 & 2 & 3 & 4 & 5 \\
      1 & 3 & 4 & 5 & 2
    \end{pmatrix}
    \begin{pmatrix}
      1 & 2 & 3 & 4 & 5 \\
      3 & 2 & 4 & 1 & 5
    \end{pmatrix},
  \]
  并将结果写成(不相交)轮换的乘积。
\end{problem}

\begin{solution}
  \[
    \begin{pmatrix}
      1 & 2 & 3 & 4 & 5 \\
      1 & 3 & 4 & 5 & 2
    \end{pmatrix}
    \begin{pmatrix}
      1 & 2 & 3 & 4 & 5 \\
      3 & 2 & 4 & 1 & 5
    \end{pmatrix} =
    \begin{pmatrix}
      1 & 2 & 3 & 4 & 5 \\
      4 & 3 & 5 & 1 & 2
    \end{pmatrix}
    = (1\; 4) (2\; 3\; 5)
  \]
\end{solution}
%%%%%%%%%%%%%%%

%%%%%%%%%%%%%%%
\begin{problem}[\score{3} $\star\star\star$]
  考虑如下定义。
  \begin{definition}[元素的阶]
    设 $G$ 是有限群, $e$ 为 $G$ 的单位元, $a \in G$。
    使 $a^{r} = e$ 成立的最小正整数称为 $a$ 的阶
    (order)~\footnote{注意, 群的阶指的是集合 $G$ 的大小, 即 $|G|$。},
    记作 $\text{ord}\; a = r$。
  \end{definition}
  设 $G$ 是有限群。请证明,
  \[
    \forall a \in G.\; (\text{ord}\; a) \big\vert |G|.
  \]
\end{problem}

\begin{proof}
  令 $\langle a \rangle = \set{a, a^{2}, a^{3}, \dots, a^{r} = e}$。
  易知, $\langle a \rangle \le G$。
  又有 $|\langle a \rangle| = \text{ord}\; a$。
  根据 Lagrange's Theorem,
  \[
    (\text{ord}\; a) \big\vert |G|.
  \]
\end{proof}
%%%%%%%%%%%%%%%

%%%%%%%%%%%%%%%
\begin{problem}[\score{5 = 2 + 1 + 2} $\star\star\star$]
  考虑从乘法群 $\R^{\ast} = \R \setminus \set{0}$
  到乘法群 $\R^{+}$ 的函数 $f: x \mapsto |x|$。
  \begin{enumerate}[(1)]
    \item 请证明, $f$ 是从 $\R^{\ast}$ 到 $\R^{+}$ 的同态。
    \item 求 $\text{Ker}\; \phi$。
    \item 套用群同态基本定理, 给出相应结论, 并用一两句话解释该结论。
  \end{enumerate}
\end{problem}

\begin{proof}
  \begin{enumerate}[(1)]
    \item
      \[
        f(xy) = |xy| = |x| |y| = f(x) f(y),
      \]
      因此 $f$ 是从 $\R^{\ast}$ 到 $\R^{+}$ 的同态。
    \item
      \[
        \text{Ker}\; \phi = \set{x \in \R^{\ast} \mid f(x) = 1} = \set{1, -1}.
      \]
    \item $\phi(\R^{\ast}) = \R^{+}$。根据群同态基本定理,
      \[
        \R^{\ast}/\set{1, -1} \cong \R^{+}.
      \]
      对于任意 $x > 0$, 陪集 $\set{x, -x}$ 被 $\phi$ 映射到 $x$。
      这说明在只考虑绝对值的情况下, $x$ 与 $-x$ 是一样的。
  \end{enumerate}
\end{proof}
%%%%%%%%%%%%%%%

%%%%%%%%%%%%%%%%%%%%
% 如果没有需要订正的题目,可以把这部分删掉
\begincorrection
%%%%%%%%%%%%%%%%%%%%

%%%%%%%%%%%%%%%%%%%%
% 如果没有反馈,可以把这部分删掉
\beginfb

你可以写 (也可以发邮件或者使用``教学立方'')
\begin{itemize}
  \item 对课程及教师的建议与意见
  \item 教材中不理解的内容
  \item 希望深入了解的内容
  \item $\cdots$
\end{itemize}
%%%%%%%%%%%%%%%%%%%%
\end{document}