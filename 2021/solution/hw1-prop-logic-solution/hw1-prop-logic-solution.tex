% hw1-prop-logic-solution.tex

% !TEX program = xelatex
%%%%%%%%%%%%%%%%%%%%
% see http://mirrors.concertpass.com/tex-archive/macros/latex/contrib/tufte-latex/sample-handout.pdf
% for how to use tufte-handout
\documentclass[a4paper, justified]{tufte-handout}

\input{hw-preamble} % feel free to modify this file if you understand LaTeX well
%%%%%%%%%%%%%%%%%%%%
\title{离散数学 (1-prop-logic)}
\me{魏恒峰}{hfwei@nju.edu.cn}{}{}
\date{2021年3月11日}
%%%%%%%%%%%%%%%%%%%%
\begin{document}
\maketitle
%%%%%%%%%%%%%%%%%%%%
\noplagiarism % PLEASE DON'T DELETE THIS LINE!
%%%%%%%%%%%%%%%%%%%%
\begin{abstract}
  % \mfigcap{width = 0.85\textwidth}{figs/George-Boole}{George Boole}
  % \begin{center}{\fcolorbox{blue}{yellow!60}{\parbox{0.65\textwidth}{\large
  %   \begin{itemize}
  %     \item
  %   \end{itemize}}}}
  % \end{center}
\end{abstract}
%%%%%%%%%%%%%%%%%%%%
\beginrequired
%%%%%%%%%%%%%%%
\begin{problem}[命题逻辑公式上的数学归纳法 \score{2 ($\star\star$)}]
  假设公式 $\alpha$ 中不含 ``$\lnot$'' 符号。
  请证明, $\alpha$ 中超过四分之一的符号是命题符号。
\end{problem}

\begin{solution}
  我们证明如下引理:
  \begin{lemma*}
    长度为 $4k + 1$ 的不含 $\lnot$ 符号的公式中含有 $k + 1$ 个命题符号。
  \end{lemma*}

  \begin{proof}
    对公式 $\alpha$ 的结构作归纳~\footnote{对公式使用数学归纳法时, 通常都是这样的做法,
    请注意以下的书写规范。其中, ``归纳假设''有时可以省略。}。

    \begin{description}
      \item[基础步骤:] $\alpha$ 是一个命题符号。
        公式长度 $|\alpha| = 1 = 4 \times 0 + 1$, 命题符号个数为 $1 = 0 + 1$。 引理成立。
      \item[归纳假设:] 长度为 $4k + 1$ 的不含 $\lnot$ 符号的公式中含有 $k + 1$ 个命题符号。
      \item[归纳步骤:] 因为$\alpha$中不含$\lnot$联词,
        所以 $\alpha$ 呈型 $\beta \oplus \gamma$,
        其中 $\oplus$ 表示 $\land$, $\lor$, $\to$ 或 $\leftrightarrow$ 二元逻辑联词。
        根据归纳假设, 对于子公式 $\beta$ 与 $\gamma$
        ($\beta_{P}$表示$\beta$中命题符号的个数),
        \[
          |\beta| = 4k_{1} + 1, \qquad |\beta_{P}| = k_{1} + 1,
        \]
        \[
          |\gamma| = 4k_{2} + 1, \qquad |\gamma_{P}| = k_{2} + 1.
        \]
        所以~\footnote{$+3$: 左括号、右括号、逻辑联词。 本解答将括号也计算在内。},
        \begin{align*}
          |\alpha| &= |\beta| + |\gamma| + 3 \\
                   &= (4k_{1} + 1) + (4k_{2} + 1) + 3 \\
                   &= 4(k_{1} + k_{2} + 1) + 1
        \end{align*}
        \begin{align*}
          |\alpha_{P}| &= |\beta_{P}| + |\gamma_{P}| \\
                   &= (k_{1} + 1) + (k_{2} + 1) \\
                   &= (k_{1} + k_{2} + 1) + 1
        \end{align*}
        得证。
    \end{description}
  \end{proof}
\end{solution}
%%%%%%%%%%%%%%%

%%%%%%%%%%%%%%%
\begin{problem}[合取范式与析取范式 \score{3 ($\star$)}]
  我们先引入一个定义。

  \begin{definition}[合取范式 (Conjunctive Normal Form; CNF)]
    \label{def:cnf}
    我们称公式 $\alpha$ 是\red{\bf 合取范式}, 如果它形如
    \[
      \alpha = \beta_{1} \land \beta_{2} \land \dots \land \beta_{k},
    \]
    其中, 每个 $\beta_{i}$ 都形如
    \[
      \beta_{i} = \beta_{i1} \lor \beta_{i2} \lor \dots \lor \beta_{in},
    \]
    并且 $\beta_{ij}$ 或是一个命题符号, 或是命题符号的否定。
  \end{definition}

  例如, 下面的公式就是一个合取范式。
  \[
    (P \lor \lnot Q \lor R) \;\red{\land}\; (\lnot P \lor Q) \;\red{\land}\; \lnot Q
  \]

  将定义~\ref{def:cnf} 中的所有$\land$换成$\lor$, 所有$\lor$换成$\land$, 其余不变,
  就变成了析取范式 (Disjunctive Normal Form; DNF)的定义。本题以CNF为例。

  将任意公式转化成CNF或DNF的方法如下:
  \begin{enumerate}[(1)]
    \item 先将公式中的联词化归成 $\lnot$, $\land$ 与 $\lor$;
    \item 再使用 De Morgan 律将 $\lnot$ 移到各个命题变元之前 (\blue{``否定深入''});
    \item 最后使用结合律、分配律将公式化归成合取范式或析取范式。
  \end{enumerate}

  请将
  \[
    (P \land (Q \to R)) \to S
  \]
  化为合取范式。
\end{problem}

\begin{solution}
  \marginnote{注意: 使用重言式进行公式变换, 要使用 `$\equiv$', 不能使用 `$=$'。}
  \begin{align*}
    (P \land (Q \to R)) \to S
    &\equiv (P \land (\lnot Q \lor R)) \to S \\
    &\equiv \lnot (P \land (\lnot Q \lor R)) \lor S \\
    &\equiv (\lnot P) \lor (\lnot (\lnot Q \lor R)) \lor S \\
    &\equiv (\lnot P) \lor (Q \land \lnot R) \lor S \\
    &\equiv (\lnot P \lor S) \lor (Q \land \lnot R) \\
    &\equiv (\lnot P \lor S \lor Q) \land (\lnot P \lor S \lor \lnot R)
  \end{align*}
  \marginnote{See \href{https://www.wolframalpha.com/input/?i=CNF+\%28P+\%26\%26+\%28Q+implies+R\%29\%29+implies+S}{CNF\&wolframalpha}}
\end{solution}
%%%%%%%%%%%%%%%

%%%%%%%%%%%%%%%
\begin{problem}[重言蕴含与推理规则 \score{5 = 3 + 2 ($\star\star\star$)}]
  \begin{enumerate}[(1)]
    \item 请使用真值表方法证明
      \[
        \set{P \lor Q, P \to R, Q \to S} \models S \lor R.
      \]
    \item 请使用重言式所代表的推理规则(可以任意使用规则,
      也可以使用你认为显然成立但课堂上没有列出来的规则, 但需要指明每一步使用了哪条规则)证明
      \[
        \set{P \lor Q, P \to R, Q \to S} \vdash S \lor R.
      \]
      提示: 你可能需要使用
      \[
        (\alpha \to \beta) \leftrightarrow (\lnot \alpha \lor \beta)
      \]
      \[
        ((\alpha \to \beta) \land (\beta \to \gamma)) \to (\alpha \to \gamma)
      \]
  \end{enumerate}
\end{problem}

\begin{solution}
  \begin{enumerate}[(1)]
    \item 见下表。 从表中可以看出, 使得 $P \lor Q$, $P \to R$, $Q \to S$
      同时成立的真值指派 (蓝色) 都满足 $S \lor R$ (红色), 符合重言蕴含的定义
      ~\footnote{需要给出完整的真值表, 并检查是否符合重言蕴含的定义}。
      % truth-table.tex

\newcolumntype{C}{>{$}c<{$}} % math-mode version of "c" column type

% \usepackage{graphicx}
\begin{table}[h]
  \centering
  \resizebox{0.50\textwidth}{!}{%
  \begin{tabular}{|C|C|C|C||C|C|C||C|}
    \hline
    P & Q & R & S & P \lor Q & P \to R & Q \to S & S \lor R \\ \hline
    T & T & T & T & \blue{T} & \blue{T} & \blue{T} & \red{T} \\ \hline
    T & T & T & F & T & T & F & T \\ \hline
    T & T & F & T & T & F & T & T \\ \hline
    T & T & F & F & T & F & F & F \\ \hline
    T & F & T & T & \blue{T} & \blue{T} & \blue{T} & \red{T} \\ \hline
    T & F & T & F & \blue{T} & \blue{T} & \blue{T} & \red{T} \\ \hline
    T & F & F & T & T & F & T & T \\ \hline
    T & F & F & F & T & F & T & F \\ \hline
    F & T & T & T & \blue{T} & \blue{T} & \blue{T} & \red{T} \\ \hline
    F & T & T & F & T & T & F & T \\ \hline
    F & T & F & T & \blue{T} & \blue{T} & \blue{T} & \red{T} \\ \hline
    F & T & F & F & T & T & F & F \\ \hline
    F & F & T & T & F & T & T & T \\ \hline
    F & F & T & F & F & T & T & T \\ \hline
    F & F & F & T & F & T & T & T \\ \hline
    F & F & F & F & F & T & T & F \\ \hline
  \end{tabular}}
\end{table}
      \marginnote{See also \href{https://www.wolframalpha.com/input/?i=truth+table+\%28\%28P+or+Q\%29+and+\%28P+implies+R\%29+and+\%28Q+implies+S\%29\%29+implies+\%28S+or+R\%29}{Truth Table@wolframalpha}}
    \item
      第一种解法中将使用如下重言式作为推理规则:
      \begin{align}
        (\alpha \to \beta) \leftrightarrow (\lnot \alpha \lor \beta)
          \tag{a} \label{rule:a} \\
        ((\alpha \to \beta) \land (\beta \to \gamma)) \to (\alpha \to \gamma)
          \tag{b} \label{rule:b} \\
        (\alpha \to \beta) \leftrightarrow (\lnot \beta \to \lnot \alpha)
          \tag{c} \label{rule:c}
      \end{align}
      \marginnote{
        \begin{itemize}
          \item 问: 考试时是否需要使用这样的书写格式?
          \item 答: 不一定。但是要尽量给公式编号, 并且写明每一步推理的所依赖的之前的公式的编号。
            对于不那么显然的推理步骤, 最好还要附带理由, 比如使用了什么推理规则。
        \end{itemize}
      }
      \begin{align}
        P \lor Q  &\qquad (\text{前提}) \label{eq:1} \\
        P \to R   &\qquad (\text{前提}) \label{eq:2} \\
        Q \to S   &\qquad (\text{前提}) \label{eq:3} \\
        \lnot P \to Q &\qquad (\ref{eq:1}, \ref{rule:a}) \label{eq:4} \\
        \lnot P \to S &\qquad (\ref{eq:4}, \ref{eq:3}, \ref{rule:b}) \label{eq:5} \\
        \lnot S \to P &\qquad (\ref{eq:5}, \ref{rule:c}) \label{eq:6} \\
        \lnot S \to R &\qquad (\ref{eq:6}, \ref{eq:2}, \ref{rule:b}) \label{eq:7} \\
        S \lor R  &\qquad (\ref{eq:7}, \ref{rule:a})
      \end{align}
      \marginnote{
        \begin{itemize}
          \item 问: 考试时是否只能使用纯符号进行推理?
          \item 答: 不一定。可以夹杂文字描述。只要推理过程清晰即可。
        \end{itemize}
      }

    第二种解法使用(后面介绍的) $\lor$-推理规则:
      \setcounter{equation}{0}
      \begin{align}
        P \lor Q  &\qquad (\text{前提}) \label{eq:1} \\
        P \to R   &\qquad (\text{前提}) \label{eq:2} \\
        Q \to S   &\qquad (\text{前提}) \label{eq:3} \\
        P \to (S \lor R) &\qquad (\ref{eq:2}) \label{eq:4} \\
        Q \to (S \lor R) &\qquad (\ref{eq:3}) \label{eq:5} \\
        S \lor R &\qquad (\ref{eq:1}, \ref{eq:4}, \ref{eq:5}, \lor\text{-elim; case analysis})
      \end{align}
    其中, (\ref{eq:4}) 的详细推理过程如下 ((\ref{eq:5}) 的推理过程类似):
    \marginnote{
      \begin{itemize}
        \item 问: 考试的时候是否需要写出这些详细的推理过程?
        \item 答: 不一定。只要每一步推理都是较为显然的, 就不需要写出它的细节。
      \end{itemize}}
      \begin{align}
        P \to R &\qquad (\text{前提})
          \tag{4-1} \label{eq:4-1} \\
        [P]     &\qquad (\text{引入假设})
          \tag{4-2} \label{eq:4-2} \\
        R       &\qquad (\ref{eq:4-1}, \ref{eq:4-2}, \to\text{-elim})
          \tag{4-3} \label{eq:4-3} \\
        S \lor R &\qquad (\ref{eq:4-3}, \lor\text{-intro})
          \tag{4-4} \label{eq:4-4} \\
        P \to (S \lor R) &\qquad (\ref{eq:4-2}, \ref{eq:4-3}, \ref{eq:4-4}, \to\text{-intro})
          \tag{4-5} \label{eq:4-5}
      \end{align}
  \end{enumerate}
\end{solution}
%%%%%%%%%%%%%%%

%%%%%%%%%%%%%%%%%%%%
% 如果没有需要订正的题目,可以把这部分删掉

\begincorrection
%%%%%%%%%%%%%%%%%%%%

%%%%%%%%%%%%%%%%%%%%
% 如果没有反馈,可以把这部分删掉
\beginfb

你可以写 (也可以发邮件或者使用``教学立方'')
\begin{itemize}
  \item 对课程及教师的建议与意见
  \item 教材中不理解的内容
  \item 希望深入了解的内容
  \item $\cdots$
\end{itemize}
%%%%%%%%%%%%%%%%%%%%
\end{document}