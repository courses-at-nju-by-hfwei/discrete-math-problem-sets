% hw11-planarity-coloring-solution.tex

% !TEX program = xelatex
%%%%%%%%%%%%%%%%%%%%
% see http://mirrors.concertpass.com/tex-archive/macros/latex/contrib/tufte-latex/sample-handout.pdf
% for how to use tufte-handout
\documentclass[a4paper, justified]{tufte-handout}

\input{hw-preamble} % feel free to modify this file if you understand LaTeX well
%%%%%%%%%%%%%%%%%%%%
\title{11. 图论: 平面图与图着色 (11-planarity-coloring)}
\me{魏恒峰}{hfwei@nju.edu.cn}{}{}
\date{2021年05月21日 发布作业 \\ 2021年06月14日 发布答案}
%%%%%%%%%%%%%%%%%%%%
\begin{document}
\maketitle
%%%%%%%%%%%%%%%%%%%%
\noplagiarism % PLEASE DON'T DELETE THIS LINE!
%%%%%%%%%%%%%%%%%%%%
\begin{abstract}
  % \mfigcap{width = 0.85\textwidth}{figs/George-Boole}{George Boole}
  % \begin{center}{\fcolorbox{blue}{yellow!60}{\parbox{0.65\textwidth}{\large
  %   \begin{itemize}
  %     \item
  %   \end{itemize}}}}
  % \end{center}
\end{abstract}
%%%%%%%%%%%%%%%%%%%%
\beginrequired
%%%%%%%%%%%%%%%

%%%%%%%%%%%%%%%
\begin{problem}[\score{4} $\star\star\star$]
  假设 $G$ 是顶点数 $\ge 11$ 的简单图, $\overline{G}$ 是 $G$ 的补图~\footnote{
    补图: 顶点集相同, 但是 $e$ 是 $G$ 的边当且仅当 $e$ 不是 $\overline{G}$ 的边。
  }。
  请证明, $G$ 和 $\overline{G}$ 不同为平面图。
\end{problem}

\begin{proof}
  反设 $G$ 和 $\overline{G}$ 都是平面图。因此,
  \begin{align}
    m \le 3n - 6 \\
    \overline{m} \le 3n - 6 \\
    m + \overline{m} = \frac{n(n-1)}{2}
  \end{align}
  解得~\footnote{\url{https://www.wolframalpha.com/input/?i=n\%5E2+-+13n+\%2B+24+\%3C\%3D+0}},
  \[
    \frac{13 -\sqrt{73}}{2} \le n \le \frac{13 + \sqrt{73}}{2}.
  \]
  其整数解为
  \[
    n \in \mathbb{Z} \land 3 \le n \le 10.
  \]
  与 $n \ge 11$ 矛盾。
\end{proof}
%%%%%%%%%%%%%%%

%%%%%%%%%%%%%%%
\begin{problem}[\score{4} $\star\star\star$]
  假设 $G$ 是包含 $n$ 个顶点的$d$-正则简单图。
  请证明
  \[
    \chi(G) \ge \frac{n}{n-d}.
  \]
\end{problem}

\begin{proof}
  反设
  \[
    \chi(G) < \frac{n}{n-d}.
  \]
  因此, 将 $G$ 中顶点按照颜色归类, 可以将 $G$ 划分成 $< \frac{n}{n-d}$ 组,
  每组内的顶点两两不相邻~\footnote{注, 这种组被称为``独立集''}。
  根据鸽笼原理, 至少存在一组含有 $> \frac{n}{\frac{n}{n-d}} = n - d$ 个顶点。
  由于 $G$ 是 $d$-正则简单图, 该组中必有两顶点相邻。矛盾。
\end{proof}
%%%%%%%%%%%%%%%

%%%%%%%%%%%%%%%
\begin{problem}[\score{4} $\star\star\star$]
  假设 $G$ 是不包含三角形 $\triangle$ 的简单平面图。
  \begin{enumerate}[(1)]
    \item 请使用Euler公式证明 $G$ 含有度数 $\le 3$ 的顶点。
    \item 请使用数学归纳法证明 $G$ 是 $4$-可着色的。
  \end{enumerate}
\end{problem}

\begin{proof}
  \begin{enumerate}[(1)]
    \item 反设 $\delta(G) \ge 4$。因此,
      \[
        \sum_{v} \deg(v) \ge 4n.
      \]
      由于 $G$ 是不包含$\triangle$的简单平面图,
      \[
        m \le 2n - 4.
      \]
      根据握手定理,
      \[
        \sum_{v} \deg(v) = 2m \le 4n - 8.
      \]
      矛盾。
    \item 对 $G$ 中顶点数 $n$ 作归纳。
      \begin{description}
        \item[基础步骤:] $n = 1$。$G$ 显然是 $4$-可着色的。
        \item[归纳假设:] 假设命题对包含 $1 \le k = n-1$ 个顶点的任意符合要求的图均成立。
        \item[归纳步骤:] 考虑包含 $n$ 个顶点的图 $G$。根据 (1), $G$ 中包含度数 $\le 3$
          的顶点, 记为 $v$。从 $G$ 中删除 $v$, 得到图 $G'$。$G'$ 包含 $n-1$ 个顶点,
          且是不包含$\triangle$的简单平面图。根据归纳假设, $G'$ 是 $4$-可着色的。
          考虑 $G'$ 的任意一种 $4$-着色方案, 将 $v$ (及删除的边) 加入 $G'$, 重新得到图 $G$。
          因为 $\deg(v) \le 3$, 所以 $G$ 是 $4$-可着色的。
      \end{description}
  \end{enumerate}
\end{proof}
%%%%%%%%%%%%%%%

%%%%%%%%%%%%%%%
\begin{problem}[\score{4} $\star\star$]
  假设图 $G_{1}$ 与 $G_{2}$ 是 homeomorphic 的。请证明~\footnote{
    $m$, $n$ 分别表示边数与点数。
  }:
  \[
    m_{1} - n_{1} = m_{2} - n_{2}.
  \]
\end{problem}

\begin{proof}
  分别考虑``插入''与``收缩''2度顶点的操作:
  \begin{itemize}
    \item 对于``插入''一个2度顶点, 顶点数 $n$ 加一, 边数 $m$ 加一, 所以 $m-n$ 保持不变。
    \item 对于``收缩''一个2度顶点, 顶点数 $n$ 减一, 边数 $m$ 减一, 所以 $m-n$ 保持不变。
  \end{itemize}
  对从 $G_{1}$ 转化到 $G_{2}$ 的``插入/收缩''操作序列的长度作归纳, 即得证。
  % 假设 $G_{1}$ 通过长度为 $n$ 的``插入/收缩''操作序列得到 $G_{2}$:
  % \[
  %   G_{1}^{0} = G_{1} \to G_{1}^{1} \to G_{1}^{2} \to \dots \to G_{1}^{n} = G_{2}.
  % \]
  % 我们证明一个更强的结论:
  % \[
  %   \forall 0 \le i \le j \le n.\;
  %     m_{i} - n_{i} = m_{j} - n_{j}.
  % \]
  % 对序列长度 $n$ 作归纳。
  % \begin{description}
  %   \item[基础步骤:] $n = 0$。$G_{1} = G_{2}$。结论显然成立。
  %   \item[归纳假设:] 假设结论对 $1 \le k = n - 1$ 成立。
  %   \item[归纳步骤:]
  % \end{description}
\end{proof}
%%%%%%%%%%%%%%%

%%%%%%%%%%%%%%%
\begin{problem}[\score{4} $\star\star$]
  请使用 Kuratowski 定理说明下图不是平面图~\footnote{你不需要制作\textsf{.gif}。}:
  \fig{width = 0.50\textwidth}{figs/planarity}
\end{problem}

\begin{proof}
  先去掉图内部中间的两个点, 然后收缩剩下的所有2度顶点, 得到一个 $K_{3,3}$。
  根据 Kuratowski 定理, 该图不是平面图。
\end{proof}
%%%%%%%%%%%%%%%

%%%%%%%%%%%%%%%%%%%%
% 如果没有需要订正的题目,可以把这部分删掉
\begincorrection
%%%%%%%%%%%%%%%%%%%%

%%%%%%%%%%%%%%%%%%%%
% 如果没有反馈,可以把这部分删掉
\beginfb

你可以写 (也可以发邮件或者使用``教学立方'')
\begin{itemize}
  \item 对课程及教师的建议与意见
  \item 教材中不理解的内容
  \item 希望深入了解的内容
  \item $\cdots$
\end{itemize}
%%%%%%%%%%%%%%%%%%%%
\end{document}