% hw6-function-solution.tex

% !TEX program = xelatex
%%%%%%%%%%%%%%%%%%%%
% see http://mirrors.concertpass.com/tex-archive/macros/latex/contrib/tufte-latex/sample-handout.pdf
% for how to use tufte-handout
\documentclass[a4paper, justified]{tufte-handout}

\input{hw-preamble} % feel free to modify this file if you understand LaTeX well
%%%%%%%%%%%%%%%%%%%%
\title{6. 集合: 函数 (6-function)}
\me{魏恒峰}{hfwei@nju.edu.cn}{}{}
\date{2021年04月15日 发布作业 \\ 2021年05月01日 发布答案}
%%%%%%%%%%%%%%%%%%%%
\begin{document}
\maketitle
%%%%%%%%%%%%%%%%%%%%
\noplagiarism % PLEASE DON'T DELETE THIS LINE!
%%%%%%%%%%%%%%%%%%%%
\begin{abstract}
  若有疑问, 可到 \url{https://github.com/courses-at-nju-by-hfwei/discrete-math-problem-sets/discussions}
  讨论。
  % \mfigcap{width = 0.85\textwidth}{figs/George-Boole}{George Boole}
  % \begin{center}{\fcolorbox{blue}{yellow!60}{\parbox{0.65\textwidth}{\large
  %   \begin{itemize}
  %     \item
  %   \end{itemize}}}}
  % \end{center}
\end{abstract}
%%%%%%%%%%%%%%%%%%%%
\beginrequired
%%%%%%%%%%%%%%%

%%%%%%%%%%%%%%%
\begin{problem}[等价关系 \score{3} $\star\star$]
  设 $R$ 是 $X$ 上的等价关系。请证明,
  \[
    \forall a, b \in X.\; ([a]_{R} = [b]_{R} \leftrightarrow a R b).
  \]
\end{problem}

\begin{proof}
  \begin{itemize}
    \item 先证 $[a]_{R} = [b]_{R} \implies a R b$: \\
      假设 $[a]_{R} = [b]_{R}$。
      首先, $b \in [b]_{R}$, 因此 $b \in [a]_{R} = [b]_{R}$, 故 $a R b$。
    \item 再证 $a R b \implies [a]_{R} = [b]_{R}$: \\
    假设 $a R b$。任取 $x$,
      \setcounter{equation}{0}
      \begin{align}
        & x \in [a]_{R} \\[6pt]
        \iff & a R x \\[6pt]
        \iff & b R x \qquad (\because a R b) \\[6pt]
        \iff & x \in [b]_{R}
      \end{align}
      因此, $[a]_{R} = [b]_{R}$。
  \end{itemize}
\end{proof}
%%%%%%%%%%%%%%%

%%%%%%%%%%%%%%%
\begin{problem}[函数与等价关系 \score{7 = 3 + 4} $\star\star\star$]
  设 $f: X \to Y$ 是满射。
  定义 $X$ 上的二元关系 $R$ 为 $(x, y) \in R$ 当且仅当 $f(x) = f(y)$。
  请证明,
  \begin{enumerate}[(1)]
    \item $R$ 是 $X$ 上的等价关系。
    \item 定义 $h \subseteq (X/R) \times Y$ 为 $h([x]_{R}) = f(x)$。
      请证明, $h$ 是从商集 $X/R$ 到 $Y$ 的函数, 且是满射。
  \end{enumerate}
\end{problem}

\begin{proof}
  \begin{enumerate}[(1)]
    \item 只需证
      \begin{itemize}
        \item $R$ 是自反的。\\
          任取 $x \in X$,
          \[
            f(x) = f(x) \implies (x, x) \in R.
          \]
        \item $R$ 是对称的。\\
          任取 $x_{1}, x_{2} \in X$,
          \begin{align*}
            & (x_{1}, x_{2}) \in R \\
            \implies & f(x_{1}) = f(x_{2}) \\
            \implies & f(x_{2}) = f(x_{1}) \\
            \implies & (x_{2}, x_{1}) \in R.
          \end{align*}
        \item $R$ 是传递的。\\
          任取 $x_{1}, x_{2}, x_{3} \in X$,
          \begin{align*}
            & (x_{1}, x_{2}) \in R \land (x_{2}, x_{3}) \in R \\
            \implies & f(x_{1}) = f(x_{2}) \land f(x_{2}) = f(x_{3}) \\
            \implies & f(x_{1}) = f(x_{3}) \\
            \implies & (x_{1}, x_{3}) \in R.
          \end{align*}
      \end{itemize}
    \item 要证 $h$ 是函数, 只需证
      \begin{itemize}
        \item $\forall S \in X/R.\; \exists y \in Y.\; h(S) = y$。\\
          设 $X/R = [x]_{R}$。取 $y = f(x)$ 即可。
        \item $\forall S \in X/R.\; \exists y_{1}, y_{2} \in Y.\;
          ((h(S) = y_{1} \land h(S) = y_{2}) \implies y_{1} = y_{2})$。\\
          根据 $h$ 的定义,
          \[
            \exists x_{1} \in X.\; ([x_{1}]_{R} = S \land f(x_{1}) = y_{1}),
          \]
          \[
            \exists x_{2} \in X.\; ([x_{2}]_{R} = S \land f(x_{2}) = y_{2}).
          \]
          因此,
          \begin{align*}
            & [x_{1}]_{R} = [x_{2}]_{R} \\
            \implies & (x_{1}, x_{2}) \in R \\
            \implies & f(x_{1}) = f(x_{2}) \qquad (\because \text{第一题结论}) \\
            \implies & y_{1} = y_{2}
          \end{align*}
      \end{itemize}
      要证 $h$ 是满射, 只需证 $\forall y \in Y.\; \exists S \in X/R.\; h(S) = y$。\\
      因为 $f$ 是满射, 对于任意 $y \in Y$, 存在 $x_{0}$, 使得 $f(x_{0}) = y$。
      故, 取 $S = [x_{0}]_{R}$ 即可。
  \end{enumerate}
\end{proof}
%%%%%%%%%%%%%%%
%%%%%%%%%%%%%%%%%%%%
% 如果没有需要订正的题目,可以把这部分删掉

\begincorrection
%%%%%%%%%%%%%%%%%%%%

%%%%%%%%%%%%%%%%%%%%
% 如果没有反馈,可以把这部分删掉
\beginfb

你可以写 (也可以发邮件或者使用``教学立方'')
\begin{itemize}
  \item 对课程及教师的建议与意见
  \item 教材中不理解的内容
  \item 希望深入了解的内容
  \item $\cdots$
\end{itemize}
%%%%%%%%%%%%%%%%%%%%
\end{document}