% hw5-relation-solution.tex

% !TEX program = xelatex
%%%%%%%%%%%%%%%%%%%%
% see http://mirrors.concertpass.com/tex-archive/macros/latex/contrib/tufte-latex/sample-handout.pdf
% for how to use tufte-handout
\documentclass[a4paper, justified]{tufte-handout}

% hw-preamble.tex

% geometry for A4 paper
% See https://tex.stackexchange.com/a/119912/23098
\geometry{
  left=20.0mm,
  top=20.0mm,
  bottom=20.0mm,
  textwidth=130mm, % main text block
  marginparsep=5.0mm, % gutter between main text block and margin notes
  marginparwidth=50.0mm % width of margin notes
}

% for colors
\usepackage{xcolor} % usage: \color{red}{text}
% predefined colors
\newcommand{\red}[1]{\textcolor{red}{#1}} % usage: \red{text}
\newcommand{\blue}[1]{\textcolor{blue}{#1}}
\newcommand{\teal}[1]{\textcolor{teal}{#1}}

\usepackage{todonotes}

% heading
\usepackage{sectsty}
\setcounter{secnumdepth}{2}
\allsectionsfont{\centering\huge\rmfamily}

% for Chinese
\usepackage{xeCJK}
\usepackage{zhnumber}
\setCJKmainfont[BoldFont=FandolSong-Bold.otf]{FandolSong-Regular.otf}

% for fonts
\usepackage{fontspec}
\newcommand{\song}{\CJKfamily{song}}
\newcommand{\kai}{\CJKfamily{kai}}

% To fix the ``MakeTextLowerCase'' bug:
% See https://github.com/Tufte-LaTeX/tufte-latex/issues/64#issuecomment-78572017
% Set up the spacing using fontspec features
\renewcommand\allcapsspacing[1]{{\addfontfeature{LetterSpace=15}#1}}
\renewcommand\smallcapsspacing[1]{{\addfontfeature{LetterSpace=10}#1}}

% for url
\usepackage{hyperref}
\hypersetup{colorlinks = true,
  linkcolor = teal,
  urlcolor  = teal,
  citecolor = blue,
  anchorcolor = blue}

\newcommand{\me}[4]{
    \author{
      {\bfseries 姓名:}\underline{#1}\hspace{2em}
      {\bfseries 学号:}\underline{#2}\hspace{2em}\\[10pt]
      {\bfseries 评分:}\underline{#3\hspace{3em}}\hspace{2em}
      {\bfseries 评阅:}\underline{#4\hspace{3em}}
  }
}

% Please ALWAYS Keep This.
\newcommand{\noplagiarism}{
  \begin{center}
    \fbox{\begin{tabular}{@{}c@{}}
      请独立完成作业,不得抄袭。\\
      若得到他人帮助, 请致谢。\\
      若参考了其它资料,请给出引用。\\
      鼓励讨论,但需独立书写解题过程。
    \end{tabular}}
  \end{center}
}

% \newcommand{\goal}[1]{
%   \begin{center}{\fcolorbox{blue}{yellow!60}{\parbox{0.50\textwidth}{\large
%     \begin{itemize}
%       \item 体会``思维的乐趣''
%       \item 初步了解递归与数学归纳法
%       \item 初步接触算法概念与问题下界概念
%     \end{itemize}}}}
%   \end{center}
% }

% Each hw consists of four parts:
\newcommand{\beginrequired}{\hspace{5em}\section{作业 (必做部分)}}
\newcommand{\beginoptional}{\section{作业 (选做部分)}}
\newcommand{\beginot}{\section{Open Topics}}
\newcommand{\begincorrection}{\section{订正}}
\newcommand{\beginfb}{\section{反馈}}

% for math
\usepackage{amsmath, mathtools, amsfonts, amssymb}
\newcommand{\set}[1]{\{#1\}}

% define theorem-like environments
\usepackage[amsmath, thmmarks]{ntheorem}

\theoremstyle{break}
\theorempreskip{2.0\topsep}
\theorembodyfont{\song}
\theoremseparator{}
\newtheorem{problem}{题目}[subsection]
\renewcommand{\theproblem}{\arabic{problem}}
\newtheorem{definition}{定义}[subsection]
\renewcommand{\thedefinition}{\arabic{definition}}
\newtheorem{lemma}{引理}[subsection]
\renewcommand{\thelemma}{\arabic{lemma}}
\newtheorem{ot}{Open Topics}

\theorempreskip{3.0\topsep}
\theoremheaderfont{\kai\bfseries}
\theoremseparator{:}
\theorempostwork{\bigskip\hrule}
\newtheorem*{solution}{解答}
\theorempostwork{\bigskip\hrule}
\newtheorem*{revision}{订正}

\theoremstyle{plain}
\newtheorem*{cause}{错因分析}
\newtheorem*{remark}{注}

\theoremstyle{break}
\theorempostwork{\bigskip\hrule}
\theoremsymbol{\ensuremath{\Box}}
\newtheorem*{proof}{证明}

% \newcommand{\ot}{\blue{\bf [OT]}}

% for figs
\renewcommand\figurename{图}
\renewcommand\tablename{表}

% for fig without caption: #1: width/size; #2: fig file
\newcommand{\fig}[2]{
  \begin{figure}[htbp]
    \centering
    \includegraphics[#1]{#2}
  \end{figure}
}
% for fig with caption: #1: width/size; #2: fig file; #3: caption
\newcommand{\figcap}[3]{
  \begin{figure}[htbp]
    \centering
    \includegraphics[#1]{#2}
    \caption{#3}
  \end{figure}
}
% for fig with both caption and label: #1: width/size; #2: fig file; #3: caption; #4: label
\newcommand{\figcaplbl}[4]{
  \begin{figure}[htbp]
    \centering
    \includegraphics[#1]{#2}
    \caption{#3}
    \label{#4}
  \end{figure}
}
% for margin fig without caption: #1: width/size; #2: fig file
\newcommand{\mfig}[2]{
  \begin{marginfigure}
    \centering
    \includegraphics[#1]{#2}
  \end{marginfigure}
}
% for margin fig with caption: #1: width/size; #2: fig file; #3: caption
\newcommand{\mfigcap}[3]{
  \begin{marginfigure}
    \centering
    \includegraphics[#1]{#2}
    \caption{#3}
  \end{marginfigure}
}

\usepackage{fancyvrb}

% for algorithms
\usepackage[]{algorithm}
\usepackage[]{algpseudocode} % noend
% See [Adjust the indentation whithin the algorithmicx-package when a line is broken](https://tex.stackexchange.com/a/68540/23098)
\newcommand{\algparbox}[1]{\parbox[t]{\dimexpr\linewidth-\algorithmicindent}{#1\strut}}
\newcommand{\hStatex}[0]{\vspace{5pt}}
\makeatletter
\newlength{\trianglerightwidth}
\settowidth{\trianglerightwidth}{$\triangleright$~}
\algnewcommand{\LineComment}[1]{\Statex \hskip\ALG@thistlm \(\triangleright\) #1}
\algnewcommand{\LineCommentCont}[1]{\Statex \hskip\ALG@thistlm%
  \parbox[t]{\dimexpr\linewidth-\ALG@thistlm}{\hangindent=\trianglerightwidth \hangafter=1 \strut$\triangleright$ #1\strut}}
\makeatother

% for footnote/marginnote
% see https://tex.stackexchange.com/a/133265/23098
\usepackage{tikz}
\newcommand{\circled}[1]{%
  \tikz[baseline=(char.base)]
  \node [draw, circle, inner sep = 0.5pt, font = \tiny, minimum size = 8pt] (char) {#1};
}
\renewcommand\thefootnote{\protect\circled{\arabic{footnote}}}

\newcommand{\score}[1]{{\bf [#1 分]}} % feel free to modify this file if you understand LaTeX well
%%%%%%%%%%%%%%%%%%%%
\title{5. 集合: 关系 (5-relation)}
\me{魏恒峰}{hfwei@nju.edu.cn}{}{}
\date{2021年04月08日 发布作业 \\ 2021年04月2x日 发布答案}
%%%%%%%%%%%%%%%%%%%%
\begin{document}
\maketitle
%%%%%%%%%%%%%%%%%%%%
\noplagiarism % PLEASE DON'T DELETE THIS LINE!
%%%%%%%%%%%%%%%%%%%%
\begin{abstract}
  若有疑问, 可到 \url{https://github.com/courses-at-nju-by-hfwei/discrete-math-problem-sets/discussions}
  讨论。
  % \mfigcap{width = 0.85\textwidth}{figs/George-Boole}{George Boole}
  % \begin{center}{\fcolorbox{blue}{yellow!60}{\parbox{0.65\textwidth}{\large
  %   \begin{itemize}
  %     \item
  %   \end{itemize}}}}
  % \end{center}
\end{abstract}
%%%%%%%%%%%%%%%%%%%%
\beginrequired
%%%%%%%%%%%%%%%

%%%%%%%%%%%%%%%
\begin{problem}[笛卡尔积 \score{3} $\star\star$]
  设 $C \neq \emptyset$, 请证明
  \[
    A \subseteq B \iff A \times C \subseteq B \times C.
  \]
\end{problem}

\begin{proof}
  先证
  \[
    A \subseteq B \implies A \times C \subseteq B \times C.
  \]
  假设 $A \subseteq B$。
  对任意有序对 $(a, c)$,
  \begin{align}
    & (a, c) \in A \times C \\[6pt]
    \iff & a \in A \land c \in C \\[6pt]
    \implies & a \in B \land c \in C \qquad (\because A \subseteq B)\\[6pt]
    \iff & (a, c) \in B \times C
  \end{align}

  再证
  \[
    A \times C \subseteq B \times C \implies A \subseteq B.
  \]
  因为 $C \neq \emptyset$, 不妨设 $c \in C$。
  假设 $A \times C \subseteq B \times C$。
  任取 $a \in A$,
  \setcounter{equation}{0}
  \begin{align}
    & a \in A \\[6pt]
    \implies &(a, c) \in A \times C \\[6pt]
    \implies &(a, c) \in B \times C
      \qquad (\because A \times C \subseteq B \times C) \\[6pt]
    \implies & a \in B \land c \in C \\[6pt]
    \implies &a \in B
  \end{align}
\end{proof}
%%%%%%%%%%%%%%%

%%%%%%%%%%%%%%%
\begin{problem}[关系的运算 \score{4} $\star\star$]
  请证明,
  \[
    R[X_1 \setminus X_2] \supseteq R[X_1] \setminus R[X_2].
  \]
  请举例说明 $\supseteq$ 不能替换成 $=$。
\end{problem}

\begin{proof}
  任取 $y$,
  \setcounter{equation}{0}
  \begin{align}
    & y \in R[X_{1}] \setminus R[X_{2}] \\[6pt]
    \iff & y \in R[X_{1}] \land y \notin R[X_{2}] \\[6pt]
    \iff & (\exists x_{1} \in X_{1}.\; (x_{1}, y) \in R)
      \land (\forall x_{2} \in X_{2}.\; (x_{2}, y) \notin R)
      \label{eq:2-3} \\[6pt]
    \implies & \exists x \in X_{1} \setminus X_{2}.\; (x, y) \in R. \\[6pt]
      \label{eq:2-4}
    \iff & y \in R[X_{1} \setminus X_{2}]
  \end{align}
  其中, 在第 (\ref{eq:2-4}) 步可取使得 (\ref{eq:2-3})
  中第一个合取子句成立的某个 $x \in X_{1}$。
  根据 (\ref{eq:2-4}) 中第二个合取子句, $x \notin X_{2}$。
  因此, $x \in X_{1} \setminus X_{2}$。

  \red{TODO: example}
\end{proof}
%%%%%%%%%%%%%%%

%%%%%%%%%%%%%%%
\begin{problem}[关系的运算 \score{4} $\star\star$]
  请证明,
  \[
    (X \cap Y) \circ Z \subseteq (X \circ Z) \cap (Y \circ Z).
  \]
  请举例说明, $\subseteq$ 不能换成 $=$。
\end{problem}

\begin{proof}
  任取 $(a, c)$,
  \setcounter{equation}{0}
  \begin{align}
    & (a, c) \in (X \cap Y) \circ Z \\[6pt]
    \iff & \exists b.\; (a, b) \in Z \land (b, c) \in X \cap Y \\[6pt]
    \iff & \exists b.\; (a, b) \in Z \land (b, c) \in X \land (b, c) \in Y \\[6pt]
    \implies & (\exists b.\; (a, b) \in Z \land (b, c) \in X)
      \land (\exists b.\; (a, b) \in Z \land (b, c) \in Y) \\[6pt]
    \iff & (a, c) \in X \circ Z \land (a, c) \in Y \circ Z \\[6pt]
    \iff & (a, c) \in (X \circ Z) \cap (Y \circ Z)
  \end{align}

  \red{TODO: example}
\end{proof}
%%%%%%%%%%%%%%%

%%%%%%%%%%%%%%%
\begin{problem}[关系的性质 \score{4} $\star\star$]
  请证明,
  \[
    R \text{ 是对称且传递的 } \iff R = R^{-1} \circ R
  \]
\end{problem}

\begin{proof}
  先证
  \[
    R \text{ 是对称且传递的 } \implies R = R^{-1} \circ R.
  \]
  假设 $R$ 是对称且传递的。
  因为 $R$ 是对称的, 所以
  \[
    R = R^{-1}.
  \]
  因为 $R$ 是传递的, 所以
  \[
    R \circ R \subseteq R.
  \]
  故,
  \[
    R^{-1} \circ R = R \circ R \subseteq R.
  \]
  其次, 任取 $(a, b) \in R$~\footnote{\red{这一步如何用``关系代数''进行运算?
  你如果有更简洁的方法, 请告诉我。}\vspace{1em}}。
  因为 $R$ 是对称的, 所以 $(b, a) \in R$。
  又因为 $R$ 是传递的, 所以 $(b, b) \in R$ 且 $(b, b) \in R^{-1}$。
  因此, $(a, b) \in R^{-1} \circ R$。故, $R \subseteq R^{-1} \circ R$。

  \marginnote{对于其它步骤, 你当然也可以``对任意 $\dots$'', 然后一顿操作猛如虎。
    但是, 在更高的``关系代数''层面上进行运算, 往往举重若轻、事半功倍。
  }
  再证
  \[
     R = R^{-1} \circ R \implies R \text{ 是对称且传递的 }.
  \]
  假设 $R = R^{-1} \circ R$。
  首先~\footnote{切记: $(R \circ S)^{-1} = S^{-1} \circ R^{-1}$},
  \[
    R^{-1} = (R^{-1} \circ R)^{-1} = R^{-1} \circ R = R.
  \]
  所以, $R$ 是对称的。其次~\footnote{这里用到了刚刚证明的 $R^{-1} = R$。},
  \[
    R = R^{-1} \circ R = R \circ R,
  \]
  所以, $R$ 是传递的。

  % \setcounter{equation}{0}
  % \begin{align}
  %   & (a, c) \in R^{-1} \circ R \\[6pt]
  %   \iff & \exists b.\; ((a, b) \in R \land (b, c) \in R^{-1}) \\[6pt]
  %   \iff & \exists b.\; ((a, b) \in R \land (c, b) \in R) \\[6pt]
  %   \implies & \exists b.\; ((b, a) \in R \land (c, b) \in R) \\[6pt]
  %   \implies & (c, a) \in R \\[6pt]
  %   \implies & (a, c) \in R
  % \end{align}
  % 因此, $R^{-1} \circ R \subseteq R$。

  % 首先, 对于任意 $(a, b)$,
  % \setcounter{equation}{0}
  % \begin{align}
  %   & (a, b) \in R \\[6pt]
  %   \implies & (a, b) \in R^{-1} \circ R \\[6pt]
  %   \implies & (b, a) \in (R^{-1} \circ R)^{-1} \\[6pt]
  %   \implies & (b, a) \in R^{-1} \circ R \\[6pt]
  %   \implies & (b, a) \in R
  % \end{align}
  % 故, $R$ 是对称的。

  % 其次, 对于任意 $(a, b)$ 与 $(b, c)$,
  % \setcounter{equation}{0}
  % \begin{align}
  %   & (a, b) \in R \land (b, c) \in R \\[6pt]
  %   \implies & (b, a) \in R \land (c, b) \in R^{-1} \\[6pt]
  %   \implies & (c, a) \in R^{-1} \circ R \\[6pt]
  %   \implies & (c, a) \in R \\[6pt]
  %   \implies & (a, c) \in R
  % \end{align}
  % 故, $R$ 是传递的。
\end{proof}
%%%%%%%%%%%%%%%

%%%%%%%%%%%%%%%
\begin{problem}[等价关系 \score{5} $\star\star\star$]
  一个自反且传递的二元关系 $R \subseteq X \times X$
  称为 $X$ 上的拟序。
  现令 $\preceq\; \subseteq X \times X$ 为拟序。
  如下定义 $X$ 上的关系 $\sim$:
  \[
    x \sim y \iff x \preceq y \land y \preceq x,
  \]
  请证明, $\sim$ 是 $X$ 上的等价关系。
\end{problem}

\begin{proof}
  \begin{enumerate}[(1)]
    \item $\sim$ 是自反的。\\
      对于任意 $x \in X$, 因为 $\preceq$ 是自反的, 所以
      \[
        x \in X \implies x \preceq x \implies x \preceq x \land x \preceq x
        \implies x \sim x.
      \]
    \item $\sim$ 是对称的。\\
      对于任意 $(x, y) \in \;\sim$,
      \[
        x \sim y \implies x \preceq y \land y \preceq x
        \implies y \preceq x \land x \preceq y \implies y \sim x.
      \]
    \item $\sim$ 是传递的。\\
      对于任意 $(x, y) \in \;\sim$ 与 $(y, z) \in \;\sim$,
      \begin{align*}
        & x \sim y \land y \sim z \\[6pt]
        \implies & (x \preceq y \land y \preceq x) \land (y \preceq z \land z \preceq y) \\[6pt]
        \implies & (x \preceq y \land y \preceq z) \land (z \preceq y \land y \preceq x) \\[6pt]
        \implies & x \preceq z \land z \preceq x \qquad (\because \;\preceq \text{is transitive})\\[6pt]
        \implies & x \sim z
      \end{align*}
  \end{enumerate}
\end{proof}
%%%%%%%%%%%%%%%
%%%%%%%%%%%%%%%%%%%%
% 如果没有需要订正的题目,可以把这部分删掉

\begincorrection
%%%%%%%%%%%%%%%%%%%%

%%%%%%%%%%%%%%%%%%%%
% 如果没有反馈,可以把这部分删掉
\beginfb

你可以写 (也可以发邮件或者使用``教学立方'')
\begin{itemize}
  \item 对课程及教师的建议与意见
  \item 教材中不理解的内容
  \item 希望深入了解的内容
  \item $\cdots$
\end{itemize}
%%%%%%%%%%%%%%%%%%%%
\end{document}