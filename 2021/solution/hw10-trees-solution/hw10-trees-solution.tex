% hw10-trees-solution.tex

% !TEX program = xelatex
%%%%%%%%%%%%%%%%%%%%
% see http://mirrors.concertpass.com/tex-archive/macros/latex/contrib/tufte-latex/sample-handout.pdf
% for how to use tufte-handout
\documentclass[a4paper, justified]{tufte-handout}

% hw-preamble.tex

% geometry for A4 paper
% See https://tex.stackexchange.com/a/119912/23098
\geometry{
  left=20.0mm,
  top=20.0mm,
  bottom=20.0mm,
  textwidth=130mm, % main text block
  marginparsep=5.0mm, % gutter between main text block and margin notes
  marginparwidth=50.0mm % width of margin notes
}

% for colors
\usepackage{xcolor} % usage: \color{red}{text}
% predefined colors
\newcommand{\red}[1]{\textcolor{red}{#1}} % usage: \red{text}
\newcommand{\blue}[1]{\textcolor{blue}{#1}}
\newcommand{\teal}[1]{\textcolor{teal}{#1}}

\usepackage{todonotes}

% heading
\usepackage{sectsty}
\setcounter{secnumdepth}{2}
\allsectionsfont{\centering\huge\rmfamily}

% for Chinese
\usepackage{xeCJK}
\usepackage{zhnumber}
\setCJKmainfont[BoldFont=FandolSong-Bold.otf]{FandolSong-Regular.otf}

% for fonts
\usepackage{fontspec}
\newcommand{\song}{\CJKfamily{song}}
\newcommand{\kai}{\CJKfamily{kai}}

% To fix the ``MakeTextLowerCase'' bug:
% See https://github.com/Tufte-LaTeX/tufte-latex/issues/64#issuecomment-78572017
% Set up the spacing using fontspec features
\renewcommand\allcapsspacing[1]{{\addfontfeature{LetterSpace=15}#1}}
\renewcommand\smallcapsspacing[1]{{\addfontfeature{LetterSpace=10}#1}}

% for url
\usepackage{hyperref}
\hypersetup{colorlinks = true,
  linkcolor = teal,
  urlcolor  = teal,
  citecolor = blue,
  anchorcolor = blue}

\newcommand{\me}[4]{
    \author{
      {\bfseries 姓名:}\underline{#1}\hspace{2em}
      {\bfseries 学号:}\underline{#2}\hspace{2em}\\[10pt]
      {\bfseries 评分:}\underline{#3\hspace{3em}}\hspace{2em}
      {\bfseries 评阅:}\underline{#4\hspace{3em}}
  }
}

% Please ALWAYS Keep This.
\newcommand{\noplagiarism}{
  \begin{center}
    \fbox{\begin{tabular}{@{}c@{}}
      请独立完成作业,不得抄袭。\\
      若得到他人帮助, 请致谢。\\
      若参考了其它资料,请给出引用。\\
      鼓励讨论,但需独立书写解题过程。
    \end{tabular}}
  \end{center}
}

% \newcommand{\goal}[1]{
%   \begin{center}{\fcolorbox{blue}{yellow!60}{\parbox{0.50\textwidth}{\large
%     \begin{itemize}
%       \item 体会``思维的乐趣''
%       \item 初步了解递归与数学归纳法
%       \item 初步接触算法概念与问题下界概念
%     \end{itemize}}}}
%   \end{center}
% }

% Each hw consists of four parts:
\newcommand{\beginrequired}{\hspace{5em}\section{作业 (必做部分)}}
\newcommand{\beginoptional}{\section{作业 (选做部分)}}
\newcommand{\beginot}{\section{Open Topics}}
\newcommand{\begincorrection}{\section{订正}}
\newcommand{\beginfb}{\section{反馈}}

% for math
\usepackage{amsmath, mathtools, amsfonts, amssymb}
\newcommand{\set}[1]{\{#1\}}

% define theorem-like environments
\usepackage[amsmath, thmmarks]{ntheorem}

\theoremstyle{break}
\theorempreskip{2.0\topsep}
\theorembodyfont{\song}
\theoremseparator{}
\newtheorem{problem}{题目}[subsection]
\renewcommand{\theproblem}{\arabic{problem}}
\newtheorem{definition}{定义}[subsection]
\renewcommand{\thedefinition}{\arabic{definition}}
\newtheorem{lemma}{引理}[subsection]
\renewcommand{\thelemma}{\arabic{lemma}}
\newtheorem{ot}{Open Topics}

\theorempreskip{3.0\topsep}
\theoremheaderfont{\kai\bfseries}
\theoremseparator{:}
\theorempostwork{\bigskip\hrule}
\newtheorem*{solution}{解答}
\theorempostwork{\bigskip\hrule}
\newtheorem*{revision}{订正}

\theoremstyle{plain}
\newtheorem*{cause}{错因分析}
\newtheorem*{remark}{注}

\theoremstyle{break}
\theorempostwork{\bigskip\hrule}
\theoremsymbol{\ensuremath{\Box}}
\newtheorem*{proof}{证明}

% \newcommand{\ot}{\blue{\bf [OT]}}

% for figs
\renewcommand\figurename{图}
\renewcommand\tablename{表}

% for fig without caption: #1: width/size; #2: fig file
\newcommand{\fig}[2]{
  \begin{figure}[htbp]
    \centering
    \includegraphics[#1]{#2}
  \end{figure}
}
% for fig with caption: #1: width/size; #2: fig file; #3: caption
\newcommand{\figcap}[3]{
  \begin{figure}[htbp]
    \centering
    \includegraphics[#1]{#2}
    \caption{#3}
  \end{figure}
}
% for fig with both caption and label: #1: width/size; #2: fig file; #3: caption; #4: label
\newcommand{\figcaplbl}[4]{
  \begin{figure}[htbp]
    \centering
    \includegraphics[#1]{#2}
    \caption{#3}
    \label{#4}
  \end{figure}
}
% for margin fig without caption: #1: width/size; #2: fig file
\newcommand{\mfig}[2]{
  \begin{marginfigure}
    \centering
    \includegraphics[#1]{#2}
  \end{marginfigure}
}
% for margin fig with caption: #1: width/size; #2: fig file; #3: caption
\newcommand{\mfigcap}[3]{
  \begin{marginfigure}
    \centering
    \includegraphics[#1]{#2}
    \caption{#3}
  \end{marginfigure}
}

\usepackage{fancyvrb}

% for algorithms
\usepackage[]{algorithm}
\usepackage[]{algpseudocode} % noend
% See [Adjust the indentation whithin the algorithmicx-package when a line is broken](https://tex.stackexchange.com/a/68540/23098)
\newcommand{\algparbox}[1]{\parbox[t]{\dimexpr\linewidth-\algorithmicindent}{#1\strut}}
\newcommand{\hStatex}[0]{\vspace{5pt}}
\makeatletter
\newlength{\trianglerightwidth}
\settowidth{\trianglerightwidth}{$\triangleright$~}
\algnewcommand{\LineComment}[1]{\Statex \hskip\ALG@thistlm \(\triangleright\) #1}
\algnewcommand{\LineCommentCont}[1]{\Statex \hskip\ALG@thistlm%
  \parbox[t]{\dimexpr\linewidth-\ALG@thistlm}{\hangindent=\trianglerightwidth \hangafter=1 \strut$\triangleright$ #1\strut}}
\makeatother

% for footnote/marginnote
% see https://tex.stackexchange.com/a/133265/23098
\usepackage{tikz}
\newcommand{\circled}[1]{%
  \tikz[baseline=(char.base)]
  \node [draw, circle, inner sep = 0.5pt, font = \tiny, minimum size = 8pt] (char) {#1};
}
\renewcommand\thefootnote{\protect\circled{\arabic{footnote}}}

\newcommand{\score}[1]{{\bf [#1 分]}} % feel free to modify this file if you understand LaTeX well
%%%%%%%%%%%%%%%%%%%%
\title{10. 图论: 树 (10-trees)}
\me{魏恒峰}{hfwei@nju.edu.cn}{}{}
\date{2021年05月13日 发布作业 \\ 2021年06月xx日 发布答案}
%%%%%%%%%%%%%%%%%%%%
\begin{document}
\maketitle
%%%%%%%%%%%%%%%%%%%%
\noplagiarism % PLEASE DON'T DELETE THIS LINE!
%%%%%%%%%%%%%%%%%%%%
\begin{abstract}
  % \mfigcap{width = 0.85\textwidth}{figs/George-Boole}{George Boole}
  % \begin{center}{\fcolorbox{blue}{yellow!60}{\parbox{0.65\textwidth}{\large
  %   \begin{itemize}
  %     \item
  %   \end{itemize}}}}
  % \end{center}
\end{abstract}
%%%%%%%%%%%%%%%%%%%%
\beginrequired
%%%%%%%%%%%%%%%

%%%%%%%%%%%%%%%
\begin{problem}[\score{4} $\star\star$]
  设 $T$ 是树且每个顶点的度数要么为1, 要么为 $k$。
  请证明~\footnote{我们经常使用 $n(G)$ 表示 $G$ 的顶点数,
    很多时候也简写为 $n$。}~\footnote{提示:
  关于顶点度数, 我们有什么定理可用?}:
  \[
    n(T) = \ell (k-1) + 2, \quad \text{for some } \ell \in \mathbb{N}.
  \]
\end{problem}

\begin{proof}
  设度数为 $k$ 的顶点数为 $m$, 则
  \[
    mk + (n-m) = 2n - 2.
  \]
  化简得,
  \[
    n = m(k - 1) + 2.
  \]
  得证。
\end{proof}
%%%%%%%%%%%%%%%

%%%%%%%%%%%%%%%
\begin{problem}[\score{4} $\star\star\star$]
  给定无向图 $G$。
  请证明: $G$ 是树当且仅当 $G$ 没有 loop 且 $G$ 有唯一的生成树。
\end{problem}

\begin{proof}
  分两个方向证明。
  \begin{itemize}
    \item $\implies:$ 假设 $G$ 是树。显然, $G$ 没有 loop。
      反设 $G$ 有两个不同的生成树 $T_{1}$, $T_{2}$。
      $T_{2}$ 中至少存在一条不在 $T_{1}$ 中的边, 记为 $e$。
      因此, $G$ 包含 $T_{1}$ 与 $e$。故 $G$ 包含圈。
      与 $G$ 是树矛盾。
    \item $\Longleftarrow:$ 假设 $G$ 没有 loop 且有唯一的生成树。
      下证
      \begin{itemize}
        \item $G$ 是连通的。因为 $G$ 有生成树, 故 $G$ 是连通的。
        \item $G$ 是无圈的。反设 $G$ 中有圈, 记为 $C$。
          设 $T$ 是 $G$ 的唯一的生成树。
          $C$ 中必存在一条不在 $T$ 中的边, 记为 $e$。
          将 $e$ 加入 $T$ 中, 必形成圈, 记为 $C'$。
          从 $C'$ 中删掉一条边 $e' \neq e$,
          得到 $G$ 的一个生成树 $T' = T + e - e'$。
          $T' \neq T$, 与 $T$ 的唯一性矛盾。
      \end{itemize}
  \end{itemize}
\end{proof}
%%%%%%%%%%%%%%%

%%%%%%%%%%%%%%%
\begin{problem}[\score{4} $\star\star\star$]
  给定无向连通图 $G$ 与 $G$ 中的某条边 $e$。
  请证明: $e$ 是桥 (bridge~\footnote{bridge 也称为 cut-edge (割边)。})
    当且仅当 $e$ 属于 $G$ 的每个生成树。
\end{problem}

\begin{proof}
  分两个方向证明。
  \begin{itemize}
    \item $\implies:$ 假设 $e$ 是桥。
      反设 $e$ 不属于 $G$ 的某个生成树 $T$。
      将 $e$ 加入 $T$ 中, 则形成一个包含 $e$ 的圈。
      所以, $e$ 不是桥。矛盾。
    \item $\Longleftarrow:$ 假设 $e$ 属于 $G$ 的每个生成树。
      反设 $e$ 不是桥, 则 $e$ 在某个圈中。
      因此, 存在某个生成树不包含 $e$
      (否则, 使用类似 Cycle Property 的证明, 可以构造出不包含 $e$ 的生成树)。
      矛盾。
  \end{itemize}
\end{proof}
%%%%%%%%%%%%%%%

%%%%%%%%%%%%%%%
\begin{problem}[\score{4 = 2 + 2} $\star\star$]
  请分别使用 Kruskal 算法与 Prim 算法 (从顶点1开始)
  给出下图的最小生成树~\footnote{以后你会明白, Kruskal
  算法与Prim算法的难度不在算法本身, 而在于搞清楚哪个是哪个。}
  要求给出边添加的顺序 (在有多种选择时, 优先选择编号较小的顶点)。

  \fig{width = 0.50\textwidth}{figs/st-example}
\end{problem}

\begin{proof}
  \begin{itemize}
    \item Kruskal 算法: 加边顺序为
      \[
        \set{1, 4}, \set{1, 2}, \set{2, 3}, \set{1, 5}, \set{2, 6}.
      \]
    \item Prim 算法: 加边顺序为~\footnote{我没注意到会是一样的}
      \[
        \set{1, 4}, \set{1, 2}, \set{2, 3}, \set{1, 5}, \set{2, 6}.
      \]
  \end{itemize}
\end{proof}
%%%%%%%%%%%%%%%

%%%%%%%%%%%%%%%
\begin{problem}[\score{4} $\star\star\star\star$]
  设 $G$ 是无向连通带权图, $T$ 是 $G$ 的一个最小生成树。

  \noindent 请证明: $T$ 是 $G$ 的唯一最小生成树当且仅当
  对于不在 $T$ 中的每一条边 $e$,
  $e$ 的权重大于 $T + e$ 所产生的圈中其它每条边的权重。
\end{problem}

\begin{proof}
  分两个方向证明。
  \begin{itemize}
    \item $\implies:$ 假设 $T$ 是 $G$ 的唯一最小生成树。
      反设\blue{存在}一条不在 $T$ 中的边 $e$,
      $w(e)$ \blue{不大于} $T + e$ 所产生的圈中其它\blue{某条边} $e'$ 的权重。
      则 $T' = T + e - e'$ 是一个权重不大于 $w(T)$ 的生成树。
      与 $T$ 的唯一性矛盾。
    \item $\Longleftarrow:$ 反设 $G$ 还有一棵最小生成树 $T' \neq T$。
      $T'$ 中至少存在一条不在 $T$ 中的边, 记为 $e$。
      将 $e$ 加入 $T$ 中, 形成一个圈 $C$。
      根据前提条件, $e$ 是 $C$ 中权重最大的唯一一条边。
      根据 Cycle Property, $e$ 不在任何最小生成树中。
      这与 $e \in T'$ 矛盾。
  \end{itemize}
\end{proof}
%%%%%%%%%%%%%%%

%%%%%%%%%%%%%%%
\begin{problem}[\score{$-10$}]
  \fig{width = 0.70\textwidth}{figs/Bourne}
\end{problem}

\begin{solution}
  不看的后果会很严重: 省下不少时间。
\end{solution}
%%%%%%%%%%%%%%%

%%%%%%%%%%%%%%%%%%%%
% 如果没有需要订正的题目,可以把这部分删掉

\begincorrection
%%%%%%%%%%%%%%%%%%%%

%%%%%%%%%%%%%%%%%%%%
% 如果没有反馈,可以把这部分删掉
\beginfb

你可以写 (也可以发邮件或者使用``教学立方'')
\begin{itemize}
  \item 对课程及教师的建议与意见
  \item 教材中不理解的内容
  \item 希望深入了解的内容
  \item $\cdots$
\end{itemize}
%%%%%%%%%%%%%%%%%%%%
\end{document}