% hw2-predicate-logic-solution.tex

% !TEX program = xelatex
%%%%%%%%%%%%%%%%%%%%
% see http://mirrors.concertpass.com/tex-archive/macros/latex/contrib/tufte-latex/sample-handout.pdf
% for how to use tufte-handout
\documentclass[a4paper, justified]{tufte-handout}

\input{hw-preamble} % feel free to modify this file if you understand LaTeX well
%%%%%%%%%%%%%%%%%%%%
\title{2. 一阶谓词逻辑 (2-predicate-logic)}
\me{魏恒峰}{hfwei@nju.edu.cn}{}{}
\date{2021年3月18日}
%%%%%%%%%%%%%%%%%%%%
\begin{document}
\maketitle
%%%%%%%%%%%%%%%%%%%%
\noplagiarism % PLEASE DON'T DELETE THIS LINE!
%%%%%%%%%%%%%%%%%%%%
\begin{abstract}
  % \mfigcap{width = 0.85\textwidth}{figs/George-Boole}{George Boole}
  % \begin{center}{\fcolorbox{blue}{yellow!60}{\parbox{0.65\textwidth}{\large
  %   \begin{itemize}
  %     \item
  %   \end{itemize}}}}
  % \end{center}
\end{abstract}
%%%%%%%%%%%%%%%%%%%%
\beginrequired
%%%%%%%%%%%%%%%

%%%%%%%%%%%%%%%
\begin{problem}[命题逻辑: 形式化描述与推理 \score{3} $\star\star$]
  张三说李四在说谎, 李四说王五在说谎, 王五说张三、李四都在说谎。
  请问, 这三人到底谁在说真话, 谁在说谎?
  \noindent {(要求: 需给出关键的推理步骤或理由)}
\end{problem}

\begin{solution}
\end{solution}
%%%%%%%%%%%%%%%

%%%%%%%%%%%%%%%
\begin{problem}[一阶谓词逻辑: 形式化描述与推理\score{3} $\star\star$]
  给定如下``前提'', 请判断``结论''是否有效, 并说明理由。
  请使用一阶谓词逻辑的知识解答。
  \noindent {(要求: 需给出关键的推理步骤或理由)} \\[10pt]

  {\bf 前提:}
  \begin{enumerate}[(1)]
    \item 每个人或者喜欢美剧, 或者喜欢韩剧 (可以同时喜欢二者);
    \item 任何人如果他喜欢抗日神剧, 他就不喜欢美剧;
    \item 有的人不喜欢韩剧。
  \end{enumerate}

  \vspace{0.20cm}
  \noindent {\bf 结论:} 有的人不喜欢抗日神剧 ({\it 幸亏如此})。
\end{problem}

\begin{solution}
\end{solution}
%%%%%%%%%%%%%%%

%%%%%%%%%%%%%%%
\begin{problem}[一阶谓词逻辑: 形式化描述与推理 \score{4} $\star\star$]
  请使用一阶谓词逻辑公式描述以下两个定义,
  并从逻辑推理的角度说明这两种定义之间是否有强弱之分。
  \noindent {(要求: 需给出关键的推理步骤或理由)}

  A function $f$ from $\mathbb{R}$ to $\mathbb{R}$ is called
  \begin{enumerate}[(1)]
    \setlength{\itemsep}{10pt}
    \item \blue{\it pointwise continuous} (连续的) if
      for every $x \in \mathbb{R}$
      and every real number $\epsilon > 0$,
      there exists real $\delta > 0$ such that
      for every $y \in \mathbb{R}$ with $|x - y| < \delta$,
      we have that $|f(x) -  f(y)|< \epsilon$.
    \item \blue{\it uniformly continuous} (一致连续的) if
      for every real number $\epsilon > 0$,
      there exists real $\delta > 0$ such that
      for every $x, y \in \mathbb{R}$ with $|x - y| < \delta$,
      we have that $|f(x) -  f(y)|< \epsilon$.
  \end{enumerate}
\end{problem}

\begin{solution}
\end{solution}
%%%%%%%%%%%%%%%

%%%%%%%%%%%%%%%%%%%%
% 如果没有需要订正的题目,可以把这部分删掉

\begincorrection
%%%%%%%%%%%%%%%%%%%%

%%%%%%%%%%%%%%%%%%%%
% 如果没有反馈,可以把这部分删掉
\beginfb

你可以写 (也可以发邮件或者使用``教学立方'')
\begin{itemize}
  \item 对课程及教师的建议与意见
  \item 教材中不理解的内容
  \item 希望深入了解的内容
  \item $\cdots$
\end{itemize}
%%%%%%%%%%%%%%%%%%%%
\end{document}