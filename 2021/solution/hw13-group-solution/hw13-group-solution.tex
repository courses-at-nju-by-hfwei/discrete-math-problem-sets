% hw13-group-solution.tex

% !TEX program = xelatex
%%%%%%%%%%%%%%%%%%%%
% see http://mirrors.concertpass.com/tex-archive/macros/latex/contrib/tufte-latex/sample-handout.pdf
% for how to use tufte-handout
\documentclass[a4paper, justified]{tufte-handout}

\input{hw-preamble} % feel free to modify this file if you understand LaTeX well
%%%%%%%%%%%%%%%%%%%%
\title{13. 群论: 基本概念 (13-group)}
\me{魏恒峰}{hfwei@nju.edu.cn}{}{}
\date{2021年06月04日 发布作业 \\ 2021年06月23日 发布答案}
%%%%%%%%%%%%%%%%%%%%
\begin{document}
\maketitle
%%%%%%%%%%%%%%%%%%%%
\noplagiarism % PLEASE DON'T DELETE THIS LINE!
%%%%%%%%%%%%%%%%%%%%
\begin{abstract}
  % \mfigcap{width = 0.85\textwidth}{figs/George-Boole}{George Boole}
  % \begin{center}{\fcolorbox{blue}{yellow!60}{\parbox{0.65\textwidth}{\large
  %   \begin{itemize}
  %     \item
  %   \end{itemize}}}}
  % \end{center}
\end{abstract}
%%%%%%%%%%%%%%%%%%%%
\beginrequired
%%%%%%%%%%%%%%%

%%%%%%%%%%%%%%%
\begin{problem}[\score{4} $\star\star$]
  请给出以下网络的一个最大流与一个最小割。
  要求给出 Ford-Fulkerson Method 运行过程。
  \fig{width = 0.50\textwidth}{figs/network-flow}
\end{problem}

\begin{proof}
\end{proof}
%%%%%%%%%%%%%%%

%%%%%%%%%%%%%%%
\begin{problem}[\score{5 = 1 + 1 + 3} $\star\star\star\star$]
  考虑下面的定理:
  \begin{theorem}[不能告诉你名字的某个著名定理]
    设 $G = (V, E)$ 是无向连通图, $v, w \in V$ 是不同的两个顶点。
    则 $v$, $w$ 之间的边不相交的 (edge-disjoint)~\footnote{
      设 $P_{1}$, $P_{2}$ 是两条$v$、$w$间的路径。
      如果 $P_{1}$ 与 $P_{2}$ 没有公共边,
      则 $P_{1}$、$P_{2}$ 是 $v$, $w$ 之间的边不相交的路径。
    } 路径的最大条数
    等于最小$vw$-边割集~\footnote{
      设 $F \subseteq E$ 为集。
      如果 $G$ 删除 $F$ 后, $v$ 与 $w$ 不再连通,
      则称 $F$ 是 $vw$-边割集。
    }的大小。
  \end{theorem}

  \fig{width = 0.50\textwidth}{figs/edge-disjoint}

  \begin{enumerate}[(1)]
    \item 考虑图中的 $v$, $w$ 顶点。
      请给出 $v$、$w$ 间的一个最大边不相交的路径集合。
    \item 考虑图中的 $v$, $w$ 顶点。
      请给出一个最小的 $vw$-边割集。
    \item 请使用最大流-最小割定理证明上述定理~\footnote{恭喜!你刚刚证明了图论中的一个著名定理。}。
  \end{enumerate}
\end{problem}

\begin{proof}
\end{proof}
%%%%%%%%%%%%%%%

%%%%%%%%%%%%%%%
\begin{problem}[\score{3} $\star\star$]
  在整数集 $\mathbb{Z}$ 中, 规定运算 $\oplus$ 如下:
  \[
    \forall a, b \in \mathbb{Z}, a \oplus b = a + b - 2.
  \]
  请证明: $(\mathbb{Z}, \oplus)$ 构成群。
\end{problem}

\begin{proof}
  只需验证:
  \begin{description}
    \item[封闭性:] 显然 $a \oplus b \in \mathbb{Z}$。
    \item[结合性:]
      \begin{align*}
        (a \oplus b) \oplus c &= (a + b - 2) \oplus c \\
          &= a + b - 2 + c - 2 \\
          &= a + (b + c - 2) - 2 \\
          &= a \oplus (b \oplus c)
      \end{align*}
    \item[单位元:] 单位元为 2。
      \[
        a \oplus 2 = a + 2 - 2 = a = 2 \oplus a.
      \]
    \item[逆元:] $a$ 的逆元是 $4-a$。
      \[
        a \oplus (4 - a) = a + (4 - a) - 2 = 2 = (4 - a) \oplus a.
      \]
  \end{description}
\end{proof}
%%%%%%%%%%%%%%%

%%%%%%%%%%%%%%%
\begin{problem}[\score{5} $\star\star\star$]
  设 $G$ 是群。
  请证明: 如果 $\forall x \in G.\; x^2 = e$,
  则 $G$ 是交换群。
\end{problem}

\begin{proof}
  对于任意 $a, b \in G$,
  \begin{align*}
    a^{2} = e &\implies a = a^{-1}, \\
    b^{2} = e &\implies b = b^{-1}, \\
    (ab)^{2} = e &\implies ab = (ab)^{-1}.
  \end{align*}
  因此,
  \begin{align*}
    ab = (ab)^{-1} = b^{-1} a^{-1} = ba.
  \end{align*}
  因此, $G$ 是交换群。
\end{proof}
%%%%%%%%%%%%%%%

%%%%%%%%%%%%%%%
\begin{problem}[\score{3} $\star\star$]
  请求出 $3^{83}$ 的最后两位数~\footnote{\url{https://www.wolframalpha.com/input/?i=3\%5E83}}。要求给出计算过程。
\end{problem}

\begin{proof}
  首先 $(3, 100) = 1$, $\phi(100) = \phi(2^2 5^2) = 100 (1 - \frac{1}{2})(1 - \frac{1}{5}) = 40$。
  根据 Euler's Theorem,
  \[
    3^{40} = 1 \mod{100}.
  \]
  其次,
  \[
    3^{83} = 3^{2 \times 40 + 3} = (3^{40})^{2} \cdot 3^{3}.
  \]
  因此,
  \[
    3^{83} \mod{100} = 27 \mod{100}.
  \]
  即 $3^{83}$ 的最后两位数是 $27$。
\end{proof}
%%%%%%%%%%%%%%%

%%%%%%%%%%%%%%%%%%%%
% 如果没有需要订正的题目,可以把这部分删掉
\begincorrection
%%%%%%%%%%%%%%%%%%%%

%%%%%%%%%%%%%%%%%%%%
% 如果没有反馈,可以把这部分删掉
\beginfb

你可以写 (也可以发邮件或者使用``教学立方'')
\begin{itemize}
  \item 对课程及教师的建议与意见
  \item 教材中不理解的内容
  \item 希望深入了解的内容
  \item $\cdots$
\end{itemize}
%%%%%%%%%%%%%%%%%%%%
\end{document}