% hw0-overview.tex

% !TEX program = xelatex
%%%%%%%%%%%%%%%%%%%%
% see http://mirrors.concertpass.com/tex-archive/macros/latex/contrib/tufte-latex/sample-handout.pdf
% for how to use tufte-handout
\documentclass[a4paper, justified]{tufte-handout}

\input{hw-preamble} % feel free to modify this file if you understand LaTeX well
%%%%%%%%%%%%%%%%%%%%
\title{离散数学 (0-Overview)}
\me{魏恒峰}{hfwei@nju.edu.cn}{}{}
\date{2021年03月04日 发布习题 \\ 2021年04月18日 发布答案}
%%%%%%%%%%%%%%%%%%%%
\begin{document}
\maketitle
%%%%%%%%%%%%%%%%%%%%
\noplagiarism % PLEASE DON'T DELETE THIS LINE!
%%%%%%%%%%%%%%%%%%%%
\begin{abstract}
  \begin{itemize}
    \item 若有疑问, 可在 \url{https://github.com/courses-at-nju-by-hfwei/discrete-math-problem-sets/discussions/new}
      中讨论。
  \end{itemize}
  % \mfigcap{width = 0.85\textwidth}{figs/George-Boole}{George Boole}
  % \begin{center}{\fcolorbox{blue}{yellow!60}{\parbox{0.65\textwidth}{\large
  %   \begin{itemize}
  %     \item
  %   \end{itemize}}}}
  % \end{center}
\end{abstract}
%%%%%%%%%%%%%%%%%%%%
\beginrequired
%%%%%%%%%%%%%%%
\begin{problem}[防疫工作, 不能大意 \score{4}]
  近期突发一种流感,症状极其严重,受感染的学生会无可遏制地进行编程与刷题等危险行为。
  假设$n^2$位学生坐在座位按$n \times n$网格状排列的教室里。
  感染正在迅速扩散:
  \begin{itemize}
    \item 如果某学生已被感染,那么他/她就不可能痊愈了;
    \item 如果某学生至少与2个已经感染的学生座位相邻 (前、后、左、右;不包括对角),
      那么该学生也会被感染。
  \end{itemize}
  请证明: 如果初始状态有$< n$个学生感染了流感,那么至少有一个学生永远不会被感染。
\end{problem}

\begin{proof}
  将相邻受感染的区域看作一个整体, 考虑受感染区域的边界长度
  \footnote{\url{https://math.stackexchange.com/a/1829606}}。

  假设初始状态有 $< n$ 个学生感染, 则受感染区域的边界长度 $\le 4(n-1)$。
  如果所有学生都被感染, 则受感染区域的边界长度为 $4n$。
  下面我们证明, 在不断感染的过程中, 受感染区域的边界长度不会增加。
  因为 $4 (n-1) < 4n$, 所以至少有一个学生不会被感染。

  \begin{lemma*}
    在不断感染的过程中, 受感染区域的边界长度不会增加。
  \end{lemma*}

  \mfig{width = 0.80\textwidth}{figs/infection}
  根据被感染的条件, 右图中的$b$周围至少有两个感染源 (用 $\times$ 表示)。
  因此, $b$ 被感染共有三种情况。
  对这三种情况分别分析, 即知引理成立。
\end{proof}
%%%%%%%%%%%%%%%

%%%%%%%%%%%%%%%
\begin{problem}[Nim Game \score{6 = 1 + 2 + 2 + 1}]
  Nim 是一个双人游戏 (你可以在课堂上分享的 Ludii Player 里找到它)。
  游戏开始时,两人面前放着几堆石头,
  两个玩家轮流操作,每次选择从某个石堆里拿走一块或多块石头。
  最后没有石头可拿的那个玩家输掉比赛。

  \fig{width = 0.40\textwidth}{figs/nim-stone}

  本题将引导大家寻找该游戏的必胜策略。

  考虑对石头堆里的石头个数(二进制表示下的; 不足时高位补0)做异或操作($\oplus$),
  结果称为 Nim 和。

  \begin{enumerate}[(1)]
    \item 请证明: 若 Nim 和为0, 则任意一次移动都会导致 Nim 和不为 0。
    \item 请证明: 若 Nim 和不为0,则必然存在一个石头堆,它的石头数大于其它所有石头堆的 Nim 和。
      (统一在二进制或十进制下进行大小比较)
    \item 请证明: 若游戏开始时, Nim 和不为 0,则先手有必胜策略。
    \item 在只有两堆石头的情况下,请给出某玩家有必胜策略的充要条件与他/她的必胜策略。
  \end{enumerate}
\end{problem}

\begin{proof}
  \begin{enumerate}[(1)]
    \item 假设移动了第$i$堆石头, 则第 $i$ 堆石头数 $n_{i}$ 的二进制表示至少有一位发生了变化。
      所以, Nim 和不再为0。
    \item 假设 Nim 和不为0, 则 Nim 和中至少有一位为 1。
      设 Nim 和共$n$位, 且最高位的1在第 $i$ 位 (从左向右计数), 则 Nim 和可表示为
      \[
        \underbrace{0 \dots 0}_{n-i} 1 \underbrace{\dots}_{i-1}
      \]
      由于 Nim 和的第 $i$ 位为 1, 则至少存在一堆石头 (任取一个这样的石头堆, 记为 $S$),
      它的石头数的二进制表示的第 $i$ 位为 1。
      下面说明$S$的石头数 (记其二进制表示为 $n_{S}$)
      大于其它所有石头堆的 Nim 和 (记为 $n_{\overline{S}}$), 理由如下:
      \begin{itemize}
        \item 由于 Nim 和的前 $n-i$ 位为0,
          所以 $n_{S}$ 的前 $n-i$ 位与 $n_{\overline{S}}$ 的前 $n-i$ 位相同;
        \item 由于 Nim 和的第 $i$ 位为 1,
          $n_{S}$ 的第 $i$ 位为 $1$, 而 $n_{\overline{S}}$ 的第 $i$ 位为 0。
      \end{itemize}
    \item 先手的必胜策略是: 每次都选择满足 (2) 中条件的石头堆, 移动若干石头, 使得 Nim 和为0。
    \item 如果初始时两堆石头的数目不同, 则先手有必胜策略: 每次均移动数目较多的那堆石头, 使得两堆石头数目相等。
      相应地, 如果初始时两堆石头的数目相等, 则后手有必胜策略。
  \end{enumerate}
\end{proof}
%%%%%%%%%%%%%%%

%%%%%%%%%%%%%%%%%%%%
% 如果没有需要订正的题目,可以把这部分删掉

\begincorrection
%%%%%%%%%%%%%%%%%%%%

%%%%%%%%%%%%%%%%%%%%
% 如果没有反馈,可以把这部分删掉
\beginfb

你可以写 (也可以发邮件或者使用``教学立方'')
\begin{itemize}
  \item 对课程及教师的建议与意见
  \item 教材中不理解的内容
  \item 希望深入了解的内容
  \item $\cdots$
\end{itemize}
%%%%%%%%%%%%%%%%%%%%
\end{document}